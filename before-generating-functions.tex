\section{Sorting algorithms}

We study two algorithms that are based on checks between keys, those
are MERGESORT and QUICKSORT.

\subsection{MERGESORT algorithm}

The MERGESORT algorithm is independent from keys present in the
input vector and it behaves the same regardless the given input.
Let $n = 2^m$ be the length of the vector to be ordered, we define a
function $C$ which count the number of checks needed to order the
input vector. $C$ definition strictly follows the algorithmic method:
\begin{displaymath}
  C(2^m) = 2C(2^{m-1}) + 2^m
\end{displaymath}
Solving the recurrence\footnote{put here the proof} we get $C(n)
\in O(n logn)$: observe that if we use an algorithm based on checks
between keys, it isn't possible to do better than to build a ``checks tree'',
hence we've found
a lower bound for the complexity of this class of algorithms \footnote{in
  the slides of the first lecture maybe there's more material about
  this topic}.

\subsection{QUICKSORT algorithm}
The QUICKSORT algorithm depends on the distribution of the keys in the
input vector. For what follow we assume to have a probability space
$\Omega = D_n$, where $D_n$ is the set of permutation of length $n$
without repetition over $\{1,\ldots,n\}$. We focus on the simpler
variant where the pivot is chosen as the right-most
key\footnote{report here the code}.  We study the behavior of an
application to the vector $\left ( 20, 25, 7, 3, 30, 8, 41,
  18\right)$, \autoref{tab:quicksort-example} reports performed steps.
\begin{table}[ht]
  \caption{Quicksort example}
  \label{tab:quicksort-example}
  \begin{center}
    \begin{tabular}{cccccccccc}
      20 & 25 & 7 & 3 & 30 & 8 & 41 & 18 &  &  \\
      $\uparrow i$ & & & & & $\uparrow j$ & & $\uparrow pivot$ &
      $\rightarrow$ & $\{20, 41, 8\}$ \\
      8 & 25 & 7 & 3 & 30 & 20 & 41 & 18 &  &  \\
       & $\uparrow i$ & & $\uparrow j$ & &  & & $\uparrow pivot$ &
       $\rightarrow$ & $\{25, 30, 3\}$ \\
       8 & 3 & 7 & 25 & 30 & 20 & 41 & 18 &  &  \\
       &  & $\uparrow j$ & $\uparrow i$ & &  & & $\uparrow pivot$ &
       $\rightarrow$ & $\{7, 25, 7\}$ \\
       8 & 3 & 7 & 18 & 30 & 20 & 41 & 25 &  &  \\
       &  &  & $\uparrow pivot$ & &  & &  &
       $\rightarrow$ & recursion \\
    \end{tabular}
  \end{center}
\end{table}
Observe that in order to move the $pivot$ element in its final
position, it is necessary for two keys ($7, 25$) to be checked
twice against the $pivot$, moreover when the second of those checks
happens, indexes $i$ and $j$ overlapped at some time such that $j < i$
eventually holds.

Hence, given a vector of length $n$, the number of checks
performed before recurring on left and right partitions is $(n-1) + 2$,
where $n-1$ appears because the $pivot$ element isn't indexed neither with
$i$ nor with $j$.  We analyze the number of performed checks by cases:

\begin{description}
\item[worst case] when the vector is already ordered, in either one of the
  two directions, recursion works over one partition only, because the
  other one has to be empty, hence the number of checks satisfies
  the following relation:
  \begin{displaymath}
    C(n) = (n-1)+2 + C(n-1)
  \end{displaymath}
  Unfolding $C(n-1)$ and fixing $C(0) = 0$ as base case, the following holds:
  \begin{displaymath}
    \begin{split}
      C(n) &= (n+1) + C(n-1) = (n+1) + n + C(n-2) = \\
      &= (n+1) + n + (n-1) + \ldots + 2 + C(0) = \\
      &= \sum_{k=2}^{n+1}{k} + C(0) = \sum_{k=1}^{n+1}{k} -1 + C(0) =
      \frac{(n+1)(n+2)}{2} - 1
    \end{split}
  \end{displaymath}
  so $C(n) \in O(n^2)$.
\item[best case] when the partition phase puts the $pivot$ element
    in the middle then QUICKSORT recurs on balanced partitions. In this
  case it has the same complexity of MERGESORT, hence $C(n) \in O(n
  logn)$
\end{description}
We explain the average case in the following dedicated section.

\subsection{QUICKSORT: On the average number of checks}

To study this case we have to consider all
elements of $\Omega$ (recall that $w \in \Omega \rightarrow (w[i]\in
\{1,\ldots,n\}) \wedge (\forall i\not =j: w[i]\not=w[j]) \wedge length(w) = n$).
Let $j$ be the $pivot$ element, hence a generic $w$ will
have this structure:
\begin{displaymath}
  w = (C_{j-1} \quad C_{n-j} \quad j)
\end{displaymath}
where $C_k$ is a vector of length $k$. Assume that the random variable
$X =$ ``$ j \in \lbrace 1,\ldots,n \rbrace \text{ is the $pivot$ element in } w$''
is uniformly distributed, formally:
\begin{displaymath}
  \mathbb{P}\left(w\in\Omega: w[n]=j \right) =
  \frac{(n-1)!}{n!} =  \frac{1}{n}
\end{displaymath}
Our goal here is to build a function $C(n)$ which counts the average
number of checks during an execution of the algorithm given an input
vector of length $n$. In order to do that observe that every key $j
\in \{1,\ldots,n\}$ can be the $pivot$, so the following holds:
\begin{displaymath}
  C(n) = (n+1) +  \frac{1}{n}\sum_{j=1}^{n}{C(j-1) + C(n-j)}
\end{displaymath}
Observing the sum when $j$ runs:
\begin{displaymath}
  \begin{split}
    j=1 &\rightarrow C(0) + C(n-1) \\
    j=2 &\rightarrow C(1) + C(n-2) \\
    \ldots& \\
    j=n-1 &\rightarrow C(n-2) + C(1) \\
    j=n &\rightarrow C(n-1) + C(0) \\
  \end{split}
\end{displaymath}
Hence we can rewrite and manipulate:
\begin{displaymath}
  \begin{split}
    C(n) &= (n+1) + \frac{2}{n}\sum_{j=0}^{n-1}{C(j)}\\
    nC(n) &= n(n+1) + 2\sum_{j=0}^{n-1}{C(j)}
  \end{split}
\end{displaymath}
Subtract the previous $(n-1)$ term to both members:
\begin{displaymath}
  \begin{split}
    nC(n) -(n-1)C(n-1) &= n(n+1) + 2\sum_{j=0}^{n-1}{C(j)} \\
    &-\left((n-1)((n-1)+1) + 2\sum_{j=0}^{(n-1)-1}{C(j)}\right) \\
    % nC(n) -(n-1)C(n-1)
    &= n(n+1) + 2\sum_{j=0}^{n-1}{C(j)} \\
    &-n(n-1) - 2\sum_{j=0}^{n-2}{C(j)} \\
    &= n(n+1-(n-1)) + 2C(n-1) \\
    &= 2(n + C(n-1)) \\
  \end{split}
\end{displaymath}
Getting $nC(n) = 2n + (n+1)C(n-1)$, divide both member by $n(n+1)$:
\begin{displaymath}
  \begin{split}
    \frac{C(n)}{n+1}  = \frac{2}{n+1} +
    \frac{C(n-1)}{n}
  \end{split}
\end{displaymath}
Matching $A(n) = b(n) + A(n-1)$, where $A(n) = \frac{C(n)}{n+1} $
and $b(n) = \frac{2}{n+1} $. Unfolding $A(n-1)$ and fixing $C(0) = 0$:
\begin{displaymath}
  \begin{split}
    \frac{C(n)}{n+1} &= \frac{2}{n+1} + \frac{C(n-1)}{n} =
    \frac{2}{n+1} +
    \frac{2}{n} + \frac{C(n-2)}{n-1}\\
    &= \frac{2}{n+1} + \frac{2}{n} + \ldots +
    \frac{2}{3} + \frac{2}{2} + \frac{C(0)}{1}\\
    &= \frac{2}{n+1} + \frac{2}{n} + \ldots +
    \frac{2}{3} + 1\\
    &= 2\left(\frac{1}{n+1} + \frac{1}{n} + \ldots +
      \frac{1}{3}\right) + 1\\
    &= 2\left(\frac{1}{n+1} + \frac{1}{n} + \ldots +
      \frac{1}{3}\right) +2\frac{1}{2} + 2
    + 1 -2\frac{1}{2} - 2\\
    &= 2\left(\frac{1}{n+1} + \frac{1}{n} + \ldots +
      \frac{1}{3}+
      \frac{1}{2}+
      1\right) -2 \\
    &= 2(H_{n+1}-1)
  \end{split}
\end{displaymath}

Where we recognized the $(n+1)$-th harmonic number $H_{n}=\left(\frac{1}{n+1} +
  \frac{1}{n} + \ldots + \frac{1}{3}+ \frac{1}{2}+ 1\right)$
to rewrite: $$C(n) = 2(n+1)(H_{n+1}-1)$$

Now recall the asymptotic approximation $H_n \sim ln(n) + \gamma$,
so $C(n) \in O(nlogn)$.
\\

From a practical point of view when a
sorting problem is approached with the QUICKSORT algorithm, to avoid
the worst case $O(n^2)$ number of checks, it is sufficient to do one
of the following actions before starting the sorting process:
\begin{itemize}
\item shuffling the input vector and proceed with the algorithm
  described above;
\item choose the $pivot$ element at random, move it in the right-most
  position and proceed with the algorithm described above.
\end{itemize}
Each one of these two tricks requires linear time in the dimension of the
input vector and allows to work with $O(nlogn)$ number of checks, on average.

\subsection{QUICKSORT: On the average number of swaps}
Our goal here is to build a function $S(n)$ which counts the average
number of swaps during an execution of the algorithm given an input
vector of length $n$.

The recurrence for the average number of swaps satisfies the following equation:
\begin{displaymath}
  S(n) =  \frac{n-2}{6} +  \frac{1}{n} \sum_{j=1}^{n}{S(j-1) + S(n-j)}
\end{displaymath}
We develop the proof in two stages, first studying:
\begin{displaymath}
  one: \mathbb{E} \left[K_j \right]  = \frac{(j-1)(n-j)}{n-1}
\end{displaymath}
where $K_j$ is a random variable that depends on $j$, after:
\begin{displaymath}
  two: \frac{n-2}{6} = \frac{1}{n}\sum_{j=1}^{n}{
    \mathbb{E} \left[K_j \right] }
\end{displaymath}
In what follow, assume to work over a probability space $\Omega$ as
defined in the previous section, distributed uniformly.

\begin{proof}[Proof of $one$]
  Let $j \in \left \lbrace 1,\ldots,n \right\rbrace $ be the $pivot$ element,
  and let $K_j: \Omega \rightarrow \mathbb{R}$ be a random variable
  such that $K_j = s \mathbb{1}_{\lbrace w[n]=j \rbrace}$, where $s$ is
  the number of swaps performed by the partitioning phase given
  an input vector $w$ of length $n$. To better understand, $K_j$ satisfies
  the following implications:
  \begin{displaymath}
    \begin{split}
      w[n] = j &\rightarrow K_j(w) = s \\
      w[n] \not= j &\rightarrow K_j(w) = 0 \\
    \end{split}
  \end{displaymath}
  The probability to have $k$ swaps when $j$ is the $pivot$ element is:
  \begin{displaymath}
    \mathbb{P}\left(K_j = k \right) =  \frac{{{n-j}\choose{k}}
      {{j-1}\choose{k}} (n-j)! (j-1)! }{(n-1)!}
  \end{displaymath}
  We can justify the above formula in the following steps:
  \begin{itemize}
  \item we have $k$ swaps when $\left| \left \lbrace
        w_i : w_i < j \right\rbrace \right| = k$ and
    $\left| \left \lbrace
        w_i : w_i > j \right\rbrace \right| = k$
  \item we can choose $k$ keys from $\left \lbrace w_i : w_i < j
    \right\rbrace$ in ${{j-1}\choose{k}}$ ways and, for each choice,
    there are $(j-1)!$ ways to sort them;
  \item we can choose $k$ keys from $\left \lbrace w_i : w_i > j
    \right\rbrace$ in ${{n-j}\choose{k}}$ ways and, for each choice,
    there are $(n-j)!$ ways to sort them;
  \item the total number of possible permutation of $n$ keys excluding
    the $pivot$ (which is fixed in the right-most position) is
    $(n-1)!$
  \end{itemize}
  Now we study the mean of the variable $K_j$:
  \begin{displaymath}
    \mathbb{E} \left[ K_j \right] = \sum_{k \geq 0}{k \mathbb{P}\left(
        K_j = k      \right) }
  \end{displaymath}
  Using ${{n}\choose{m}} =  \frac{n!}{m!(n-m)!} $, we can rewrite:
  \begin{displaymath}
    \mathbb{P}\left(K_j = k \right) =  {{n-j}\choose{k}}
    {{j-1}\choose{k}} \frac{(n-j)! (j-1)! }{(n-1)!} =  {{n-j}\choose{k}}
    {{j-1}\choose{k}} {{n-1}\choose{j-1}}^{-1}
  \end{displaymath}
  We put the previous rewrite of $\mathbb{P}\left(K_j = k \right)$ into
  the definition of $\mathbb{E} \left[ K_j \right]$:
  \begin{displaymath}
    \mathbb{E} \left[ K_j \right] = \sum_{k \geq 0}{k \mathbb{P}\left(
        K_j = k      \right) }  = \sum_{k \geq 0}{k \frac{{{n-j}\choose{k}}
        {{j-1}\choose{k}}}{{{n-1}\choose{j-1}}}}
  \end{displaymath}
  Using the following rewrite for ${{j-1}\choose{k}}$:
  \begin{displaymath}
    {{j-1}\choose{k}} =  \frac{(j-1)!}{k!(j-1-k)!} =
    \frac{(j-1)}{k} \frac{(j-2)!}{(k-1)!(j-1-k)!} =
    \frac{(j-1)}{k}{{j-2}\choose{k-1}}
  \end{displaymath}
  And ${{n}\choose{m}} = {{n}\choose{n-m}}$ implies ${{j-2}\choose{k-1}}
  = {{j-2}\choose{j-2 -(k-1)}} = {{j-2}\choose{j -k-1}}$, then:
  \begin{displaymath}
    \begin{split}
      \mathbb{E} \left[ K_j \right] &= \sum_{k \geq 0}{k
        \mathbb{P}\left( K_j = k \right) } =
      \frac{1}{{{n-1}\choose{j-1}}} \sum_{k \geq 0}{k
        {{n-j}\choose{k}} {{j-1}\choose{k}}} \\
      &= \frac{1}{{{n-1}\choose{j-1}}} \sum_{k \geq 0}{k
        {{n-j}\choose{k}}  \frac{j-1}{k}{{j-2}\choose{j-k-1}}}
      = \frac{j-1}{{{n-1}\choose{j-1}}} \sum_{k \geq 0}{
        {{n-j}\choose{k}} {{j-2}\choose{j-k-1}}}
    \end{split}
  \end{displaymath}
  Now we recognize the Vandermonde result:
  \begin{displaymath}
    \sum_{k \geq 0}{{{r}\choose{k}} {{s}\choose{n-k}}  } =
    {{r + s}\choose{n}}
  \end{displaymath}
  which can be proved directly because the ipergeometric distribution
  has exactly the same structure, and being a distribution, it sum up
  to 1. So we use this result applying it to $\sum_{k \geq 0}{
    {{n-j}\choose{k}} {{j-2}\choose{j-k-1}}} = {{n-2}\choose{j-1}} $,
  obtaining:
  \begin{displaymath}
    \begin{split}
      \mathbb{E} \left[ K_j \right] &= \frac{j-1}{{{n-1}\choose{j-1}}}
      {{n-2}\choose{j-1}} =
      \frac{(j-1)(n-2)!(j-1)!(n-j)!}{(n-1)!(j-1)!(n-j-1)!}=\\
      &=\frac{(j-1)(n-2)!(j-1)!(n-j)(n-j-1)!}
      {(n-1)(n-2)!(j-1)!(n-j-1)!}= \frac{(j-1)(n-j)}{n-1}
    \end{split}
  \end{displaymath}
\end{proof}
Now we proceed with the second proof:
\begin{proof}[Proof of $two$]
  Let us start with:
  \begin{displaymath}
    \begin{split}
      \frac{1}{n}\sum_{j=1}^{n}{\mathbb{E} \left[K_j \right] } &=
      \frac{1}{n} \sum_{j=1}^{n}{\frac{(j-1)(n-j)}{n-1}} =
      \frac{1}{n(n-1)} \sum_{j=1}^{n}{(j-1)(n-j)}=\\
      &=\frac{1}{n(n-1)} \sum_{j=1}^{n}{(j(1+n)-j^2-n)} \\
      &=\frac{1}{n(n-1)} \left( (n+1)\sum_{j=1}^{n}{j} -
        \sum_{j=1}^{n}{j^2} -n \sum_{j=1}^{n}{1}  \right)\\
      &=\frac{1}{n(n-1)}\left( \frac{n(n+1)^2}{2} -
        \frac{n(n+1)(2n+1)}{6} - n^2 \right) = \ldots =  \frac{n-2}{6}
    \end{split}
  \end{displaymath}
\end{proof}

We are now ready to solve the main recurrence using the same strategy
for the average number of checks, fixing $S(0) = S(1) = S(2) =
0$:
\begin{displaymath}
  \begin{split}
    nS(n) - (n-1)S(n-1) &=  \frac{2n-3}{6} + 2S(n-1)\\
    \frac{S(n)}{n+1} &=  \frac{S(n-1)}{n} +  \frac{2n -3}{6n(n+1)} =
      % \frac{S(2)}{3} +
      \sum_{k=3}^{n}{ \frac{2k-3}{6k(k+1)} }
  \end{split}
\end{displaymath}
Study the general term of the summation $ \frac{2k-3}{6k(k+1)}$ and
decompose it in partial fractions using Maxima:\\
\noindent
%%%%%%%%%%%%%%%
%%% INPUT:
\begin{minipage}[t]{8ex}{\color{red}\bf
\begin{verbatim}
(%i131)
\end{verbatim}}
\end{minipage}
\begin{minipage}[t]{\textwidth}{\color{blue}
\begin{verbatim}
generalTerm:(2*k-3)/(6*k*(k-1));
pf:partfrac(generalTerm,k);
integrate(pf , k);
\end{verbatim}}
\end{minipage}
%%% OUTPUT:
\definecolor{labelcolor}{RGB}{100,0,0}
\begin{math}\displaystyle
\parbox{8ex}{\color{labelcolor}(\%o131) }
\frac{2\,k-3}{6\,\left( k-1\right) \,k}
\end{math}\\
\begin{math}\displaystyle
\parbox{8ex}{\color{labelcolor}(\%o132) }
\frac{1}{2\,k}-\frac{1}{6\,\left( k-1\right) }
\end{math}\\
\begin{math}\displaystyle
\parbox{8ex}{\color{labelcolor}(\%o133) }
\frac{\mathrm{log}\left( k\right) }{2}-\frac{\mathrm{log}\left( k-1\right) }{6}
\end{math}
%%%%%%%%%%%%%%%

Hence $S(n) \in O(n logn)$.


\subsection{Final consideration}

We've finished our treatment of sorting algorithms based on checks
between keys. We've seen theoretical results under a probability space
$\Omega$ composed by permutations of $n$ objects. This allow us to
remark that if we setup a simulation where we apply one algorithm seen
above to the entire space $\Omega$ with vectors of a given dimension
$n$, then the average number of checks and swaps must be equal to the
theoretical results obtained in the previous formulas.
