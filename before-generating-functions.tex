\section{Before generating functions}

We study two algorithms that are based on checks between keys, those
are MERGESORT and QUICKSORT.

\subsection{MERGESORT algorithm}

The MERGESORT algorithm is independent from the keys present in the
input vector and its methodology behaves always the same. Let $n =
2^m$ be the length of the vector to be ordered, we can define a
function $C$ which count the number of checks needed to order the
input vector. We define the function $C$ using the method used by the
algorithm at each step:
\begin{displaymath}
  C(2^m) = 2C(2^{m-1}) + 2^m
\end{displaymath}
Solving the recurrence\footnote{put here the proof} we obtain $C(n)
\in O(n logn)$. We can observe that if we use a method (like the ones
we're studying in this section) based on checks between keys, isn't
possible to do better than build a ``checks tree'', this allow use to
have a lower bound for the complexity of those algorithms.\footnote{in
  the slides of the first lecture maybe there's more material about
  this topic}

\subsection{QUICKSORT algorithm}
The QUICKSORT algorithm depends on the distribution of the keys in the
input vector. For what follow we assume to have a probability space
$\Omega = D_n$, where $D_n$ is the set of permutation of length $n$
without repetition over $\{1,\ldots,n\}$. We focus on the simpler
variant where the pivot is chosen as the right-most
key\footnote{report here the code}.  We study the behavior of an
application to the vector $\left ( 20, 25, 7, 3, 30, 8, 41,
  18\right)$, reporting in \autoref{tab:quicksort-example} the steps
performed.
\begin{table}[ht]
  \caption{Quicksort example}
  \label{tab:quicksort-example}
  \begin{center}
    \begin{tabular}{cccccccccc}
      20 & 25 & 7 & 3 & 30 & 8 & 41 & 18 &  &  \\ 
      $\uparrow i$ & & & & & $\uparrow j$ & & $\uparrow pivot$ &
      $\rightarrow$ & $\{20, 41, 8\}$ \\
      8 & 25 & 7 & 3 & 30 & 20 & 41 & 18 &  &  \\ 
       & $\uparrow i$ & & $\uparrow j$ & &  & & $\uparrow pivot$ &
       $\rightarrow$ & $\{25, 30, 3\}$ \\
       8 & 3 & 7 & 25 & 30 & 20 & 41 & 18 &  &  \\ 
       &  & $\uparrow j$ & $\uparrow i$ & &  & & $\uparrow pivot$ &
       $\rightarrow$ & $\{7, 25, 7\}$ \\
       8 & 3 & 7 & 18 & 30 & 20 & 41 & 25 &  &  \\ 
       &  &  & $\uparrow pivot$ & &  & &  &
       $\rightarrow$ & recursion \\
    \end{tabular}
  \end{center}
\end{table}
We can observe that in order to move the $pivot$ element in its final
position, it is necessary for two keys ($7, 25$) to be checked
twice. Hence, given a vector of length $n$, the number of checks
performed is $(n-1) + 2$ ($n-1$ because the $pivot$ isn't indexed
neither with $i$ nor with $j$).

We can analyze the number of checks performed, partitioning it in the
following cases:
\begin{description}
\item[worst case] when the vector is already ordered (in one of the
  two directions). In this case we got:
  \begin{displaymath}
    C(n) = (n-1)+2 + C(n-1)
  \end{displaymath}
  recurring only on one partition because the other have to be
  empty. We can expand the recurrence, fixing $C(0) = 0$:
  \begin{displaymath}
    \begin{split}
      C(n) &= (n+1) + C(n-1) = (n+1) + n + C(n-2) = \\
      &= (n+1) + n + (n-1) + \ldots + 2 + C(0) = \\
      &= \sum_{k=2}^{n+1}{k} + C(0) = \sum_{k=1}^{n+1}{k} -1 + C(0) =
      \frac{(n+1)(n+2)}{2} - 1
    \end{split}
  \end{displaymath}
  so $C(n) \in O(n^2)$.
\item[best case] when the partition phase puts the $pivot$ in the
  middle, hence the QUICKSORT recurs on balanced partitions. In this
  case we have the same complexity of MERGESORT, hence $C(n) \in O(n
  logn)$
\item[average case] to study this case we have to consider all
  elements of $\Omega$ (recall that $w \in \Omega \rightarrow (w[i]\in
  \{1,\ldots,n\}) \wedge (\forall i\not =j: w[i]\not=w[j])$). First of
  all we can suppose that $j$ is the $pivot$, hence a generic $w$ will
  have this structure:
  \begin{displaymath}
    w = (C_{j-1} \quad C_{n-j} \quad j)
  \end{displaymath}
  where $C_k$ is a vector of length $k$. We can consider the
  probability to have $j$ as $pivot$ considering the uniform
  distribution on $\Omega$:
  \begin{displaymath}
    \mathbb{P}\left(w\in\Omega: w[n]=j \right) =
    \frac{(n-1)!}{n!} =  \frac{1}{n} 
  \end{displaymath}
  But every keys $j \in \{1,\ldots,n\}$ can be the $pivot$, so we can
  compose:
  \begin{displaymath}
    C(n) = (n+1) +  \frac{1}{n}\sum_{j=1}^{n}{C(j-1) + C(n-j)} 
  \end{displaymath}
  Observing the sum when $j$ runs:
  \begin{displaymath}
    \begin{split}
      j=1 &\rightarrow C(0) + C(n-1) \\
      j=2 &\rightarrow C(1) + C(n-2) \\
      \ldots& \\
      j=n-1 &\rightarrow C(n-2) + C(1) \\
      j=n &\rightarrow C(n-1) + C(0) \\
    \end{split}
  \end{displaymath}
  Hence we can rewrite:
  \begin{displaymath}
    C(n) = (n+1) +  \frac{2}{n}\sum_{j=0}^{n-1}{C(j)} 
  \end{displaymath}
  Now we do some manipulation:
  \begin{displaymath}
    \begin{split}
      C(n) &= (n+1) + \frac{2}{n}\sum_{j=0}^{n-1}{C(j)}\\
      nC(n) &= n(n+1) + 2\sum_{j=0}^{n-1}{C(j)}
    \end{split}
  \end{displaymath}
  Subtract the previous $(n-1)$ term to both members:
  \begin{displaymath}
    \begin{split}
      nC(n) -(n-1)C(n-1) &= n(n+1) + 2\sum_{j=0}^{n-1}{C(j)} \\
      &-\left((n-1)((n-1)+1) + 2\sum_{j=0}^{(n-1)-1}{C(j)}\right) \\
      % nC(n) -(n-1)C(n-1)
      &= n(n+1) + 2\sum_{j=0}^{n-1}{C(j)} \\
      &-n(n-1) - 2\sum_{j=0}^{n-2}{C(j)} \\
      &= n(n+1-(n-1)) + 2C(n-1) \\
      &= 2(n + C(n-1)) \\      
    \end{split}
  \end{displaymath}
  Getting $nC(n) = 2n + (n+1)C(n-1)$, divide both member by $n(n+1)$:
  \begin{displaymath}
    \begin{split}
      \frac{C(n)}{n+1}  = \frac{2}{n+1} +
      \frac{C(n-1)}{n}
    \end{split}
  \end{displaymath}
  We arrived at a recurrence $A(n) = b(n) + A(n-1)$, where $A(n) =
  \frac{C(n)}{n+1} $ and $b(n) = \frac{2}{n+1} $. So we expand, fixing
  $C(0) = 0$:
  \begin{displaymath}
    \begin{split}
      \frac{C(n)}{n+1} &= \frac{2}{n+1} + \frac{C(n-1)}{n} =
      \frac{2}{n+1} +
      \frac{2}{n} + \frac{C(n-2)}{n-1}\\
      &= \frac{2}{n+1} + \frac{2}{n} + \ldots +
      \frac{2}{3} + \frac{2}{2} + \frac{C(0)}{1}\\
      &= \frac{2}{n+1} + \frac{2}{n} + \ldots +
      \frac{2}{3} + 1\\
      &= 2\left(\frac{1}{n+1} + \frac{1}{n} + \ldots +
        \frac{1}{3}\right) + 1\\
      &= 2\left(\frac{1}{n+1} + \frac{1}{n} + \ldots +
        \frac{1}{3}\right) +2\frac{1}{2} + 2
      + 1 -2\frac{1}{2} - 2\\
      &= 2\left(\frac{1}{n+1} + \frac{1}{n} + \ldots +
        \frac{1}{3}+
        \frac{1}{2}+
        1\right) -2 \\
      &= 2(H_{n+1}-1) 
    \end{split}
  \end{displaymath}
  Having recognized the harmonic numbers $H_{n+1}=\left(\frac{1}{n+1}
    + \frac{1}{n} + \ldots + \frac{1}{3}+ \frac{1}{2}+ 1\right)$, the
  final result is $$C(n) = 2(n+1)(H_{n+1}-1)$$.

\end{description}

 