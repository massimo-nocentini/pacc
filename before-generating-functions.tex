\section{Sorting algorithms}

We study two algorithms that are based on checks between keys, those
are MERGESORT and QUICKSORT.

\subsection{MERGESORT algorithm}

The MERGESORT algorithm is independent from the keys present in the
input vector and its methodology behaves always the same. Let $n =
2^m$ be the length of the vector to be ordered, we can define a
function $C$ which count the number of checks needed to order the
input vector. We define the function $C$ using the method used by the
algorithm at each step:
\begin{displaymath}
  C(2^m) = 2C(2^{m-1}) + 2^m
\end{displaymath}
Solving the recurrence\footnote{put here the proof} we obtain $C(n)
\in O(n logn)$. We can observe that if we use a method (like the ones
we're studying in this section) based on checks between keys, isn't
possible to do better than build a ``checks tree'', this allow use to
have a lower bound for the complexity of those algorithms.\footnote{in
  the slides of the first lecture maybe there's more material about
  this topic}

\subsection{QUICKSORT algorithm}
The QUICKSORT algorithm depends on the distribution of the keys in the
input vector. For what follow we assume to have a probability space
$\Omega = D_n$, where $D_n$ is the set of permutation of length $n$
without repetition over $\{1,\ldots,n\}$. We focus on the simpler
variant where the pivot is chosen as the right-most
key\footnote{report here the code}.  We study the behavior of an
application to the vector $\left ( 20, 25, 7, 3, 30, 8, 41,
  18\right)$, reporting in \autoref{tab:quicksort-example} the steps
performed.
\begin{table}[ht]
  \caption{Quicksort example}
  \label{tab:quicksort-example}
  \begin{center}
    \begin{tabular}{cccccccccc}
      20 & 25 & 7 & 3 & 30 & 8 & 41 & 18 &  &  \\ 
      $\uparrow i$ & & & & & $\uparrow j$ & & $\uparrow pivot$ &
      $\rightarrow$ & $\{20, 41, 8\}$ \\
      8 & 25 & 7 & 3 & 30 & 20 & 41 & 18 &  &  \\ 
       & $\uparrow i$ & & $\uparrow j$ & &  & & $\uparrow pivot$ &
       $\rightarrow$ & $\{25, 30, 3\}$ \\
       8 & 3 & 7 & 25 & 30 & 20 & 41 & 18 &  &  \\ 
       &  & $\uparrow j$ & $\uparrow i$ & &  & & $\uparrow pivot$ &
       $\rightarrow$ & $\{7, 25, 7\}$ \\
       8 & 3 & 7 & 18 & 30 & 20 & 41 & 25 &  &  \\ 
       &  &  & $\uparrow pivot$ & &  & &  &
       $\rightarrow$ & recursion \\
    \end{tabular}
  \end{center}
\end{table}
We can observe that in order to move the $pivot$ element in its final
position, it is necessary for two keys ($7, 25$) to be checked
twice. Hence, given a vector of length $n$, the number of checks
performed before recurring on left and right partitions is $(n-1) + 2$
(where $n-1$ is explained by the $pivot$ isn't indexed neither with
$i$ nor with $j$).

We can analyze the number of checks performed, partitioning it in the
following cases:
\begin{description}
\item[worst case] when the vector is already ordered (in one of the
  two directions). In this case we got:
  \begin{displaymath}
    C(n) = (n-1)+2 + C(n-1)
  \end{displaymath}
  recurring only on one partition because the other have to be
  empty. We can expand the recurrence, fixing $C(0) = 0$:
  \begin{displaymath}
    \begin{split}
      C(n) &= (n+1) + C(n-1) = (n+1) + n + C(n-2) = \\
      &= (n+1) + n + (n-1) + \ldots + 2 + C(0) = \\
      &= \sum_{k=2}^{n+1}{k} + C(0) = \sum_{k=1}^{n+1}{k} -1 + C(0) =
      \frac{(n+1)(n+2)}{2} - 1
    \end{split}
  \end{displaymath}
  so $C(n) \in O(n^2)$.
\item[best case] when the partition phase puts the $pivot$ in the
  middle, hence the QUICKSORT recurs on balanced partitions. In this
  case we have the same complexity of MERGESORT, hence $C(n) \in O(n
  logn)$
\end{description}
We explain the average case in a dedicated section.

\subsection{QUICKSORT: On the average number of checks}

To study this case we have to consider all
elements of $\Omega$ (recall that $w \in \Omega \rightarrow (w[i]\in
\{1,\ldots,n\}) \wedge (\forall i\not =j: w[i]\not=w[j])$). First of
all we can suppose that $j$ is the $pivot$, hence a generic $w$ will
have this structure:
\begin{displaymath}
  w = (C_{j-1} \quad C_{n-j} \quad j)
\end{displaymath}
where $C_k$ is a vector of length $k$. We can consider the
probability to have $j$ as $pivot$ considering the uniform
distribution on $\Omega$:
\begin{displaymath}
  \mathbb{P}\left(w\in\Omega: w[n]=j \right) =
  \frac{(n-1)!}{n!} =  \frac{1}{n} 
\end{displaymath}
Our goal here is to build a function $C(n)$ which counts the average
number of checks during an execution of the algorithm given an input
vector of length $n$. In order to do that observe that every keys $j
\in \{1,\ldots,n\}$ can be the $pivot$, we can write:
\begin{displaymath}
  C(n) = (n+1) +  \frac{1}{n}\sum_{j=1}^{n}{C(j-1) + C(n-j)} 
\end{displaymath}
Observing the sum when $j$ runs:
\begin{displaymath}
  \begin{split}
    j=1 &\rightarrow C(0) + C(n-1) \\
    j=2 &\rightarrow C(1) + C(n-2) \\
    \ldots& \\
    j=n-1 &\rightarrow C(n-2) + C(1) \\
    j=n &\rightarrow C(n-1) + C(0) \\
  \end{split}
\end{displaymath}
Hence we can rewrite:
\begin{displaymath}
  C(n) = (n+1) +  \frac{2}{n}\sum_{j=0}^{n-1}{C(j)} 
\end{displaymath}
Now we do some manipulation:
\begin{displaymath}
  \begin{split}
    C(n) &= (n+1) + \frac{2}{n}\sum_{j=0}^{n-1}{C(j)}\\
    nC(n) &= n(n+1) + 2\sum_{j=0}^{n-1}{C(j)}
  \end{split}
\end{displaymath}
Subtract the previous $(n-1)$ term to both members:
\begin{displaymath}
  \begin{split}
    nC(n) -(n-1)C(n-1) &= n(n+1) + 2\sum_{j=0}^{n-1}{C(j)} \\
    &-\left((n-1)((n-1)+1) + 2\sum_{j=0}^{(n-1)-1}{C(j)}\right) \\
    % nC(n) -(n-1)C(n-1)
    &= n(n+1) + 2\sum_{j=0}^{n-1}{C(j)} \\
    &-n(n-1) - 2\sum_{j=0}^{n-2}{C(j)} \\
    &= n(n+1-(n-1)) + 2C(n-1) \\
    &= 2(n + C(n-1)) \\      
  \end{split}
\end{displaymath}
Getting $nC(n) = 2n + (n+1)C(n-1)$, divide both member by $n(n+1)$:
\begin{displaymath}
  \begin{split}
    \frac{C(n)}{n+1}  = \frac{2}{n+1} +
    \frac{C(n-1)}{n}
  \end{split}
\end{displaymath}
We arrived at a recurrence $A(n) = b(n) + A(n-1)$, where $A(n) =
\frac{C(n)}{n+1} $ and $b(n) = \frac{2}{n+1} $. So we expand, fixing
$C(0) = 0$:
\begin{displaymath}
  \begin{split}
    \frac{C(n)}{n+1} &= \frac{2}{n+1} + \frac{C(n-1)}{n} =
    \frac{2}{n+1} +
    \frac{2}{n} + \frac{C(n-2)}{n-1}\\
    &= \frac{2}{n+1} + \frac{2}{n} + \ldots +
    \frac{2}{3} + \frac{2}{2} + \frac{C(0)}{1}\\
    &= \frac{2}{n+1} + \frac{2}{n} + \ldots +
    \frac{2}{3} + 1\\
    &= 2\left(\frac{1}{n+1} + \frac{1}{n} + \ldots +
      \frac{1}{3}\right) + 1\\
    &= 2\left(\frac{1}{n+1} + \frac{1}{n} + \ldots +
      \frac{1}{3}\right) +2\frac{1}{2} + 2
    + 1 -2\frac{1}{2} - 2\\
    &= 2\left(\frac{1}{n+1} + \frac{1}{n} + \ldots +
      \frac{1}{3}+
      \frac{1}{2}+
      1\right) -2 \\
    &= 2(H_{n+1}-1) 
  \end{split}
\end{displaymath}
Having recognized the harmonic numbers $H_{n+1}=\left(\frac{1}{n+1} +
  \frac{1}{n} + \ldots + \frac{1}{3}+ \frac{1}{2}+ 1\right)$, the
final result is $$C(n) = 2(n+1)(H_{n+1}-1)$$

In order to bound $C(n)$ we have to recall that $H_n \sim ln(n) +
\gamma$, hence $C(n) \in O(nlogn)$.

From a practical point of view, to avoid the worst case, when a
sorting problem is approached with the QUICKSORT algorithm we can
choose to do one of the following actions before starting the sorting
process:
\begin{itemize}
\item shuffling the input vector and proceed with the algorithm
  described above;
\item choose the $pivot$ element at random, move it in the right-most
  position and proceed with the algorithm described above.
\end{itemize}
Each of the two tricks require linear time in the dimension of the
input vector and allow us to use the result of the average case and
working with $O(nlogn)$ number of checks.

\subsection{QUICKSORT: On the average number of swaps}
Our goal here is to build a function $S(n)$ which counts the average
number of swaps during an execution of the algorithm given an input
vector of length $n$.

The recurrence for the average number of swaps is as follow:
\begin{displaymath}
  S(n) =  \frac{n-2}{6} +  \frac{1}{n} \sum_{j=1}^{n}{S(j-1) + S(n-j)} 
\end{displaymath}
We proceed the proof in two stages, first studying
\begin{equation}
  \label{eq:quicksort-swaps-one}
  \mathbb{E} \left[K_j \right]  = \frac{(j-1)(n-j)}{n-1} 
\end{equation}
where $K_j$ is a random variable that depends on $j$, after we'll
proof:
\begin{equation}
  \label{eq:quicksort-swaps-two}
  \frac{n-2}{6} = \frac{1}{n}\sum_{j=1}^{n}{
    \mathbb{E} \left[K_j \right] }
\end{equation}
In what follow, suppose to have a probability space $\Omega$ as
defined in the previous sections and an uniform distribution above it.
\begin{proof}[Proof of \autoref{eq:quicksort-swaps-one}]
  Again, suppose the $pivot$ is $j \in \left \lbrace
    1,\ldots,n \right\rbrace $. We define the random variable $K_j:
  \Omega \rightarrow \mathbb{R}$ such that:
  \begin{displaymath}
    \begin{split}
      w[n] = j &\rightarrow K_j(w) = s \\
      w[n] \not= j &\rightarrow K_j(w) = 0 \\
    \end{split}
  \end{displaymath}
  where $s$ is the number of swaps performed by the partitioning phase
  of QUICKSORT given an input vector $w$ of length $n$ (hence the
  variable $K_j$ counts the number of swaps). Now we can compute the
  probability to have $k$ swaps when $j$ is the $pivot$:
  \begin{displaymath}
    \mathbb{P}\left(K_j = k \right) =  \frac{{{n-j}\choose{k}}
      {{j-1}\choose{k}} (n-j)! (j-1)! }{(n-1)!} 
  \end{displaymath}
  We can justify the above formula in the following steps:
  \begin{itemize}
  \item we have $k$ swaps when $\left| \left \lbrace
        w_i : w_i < j \right\rbrace \right| = k$ and
    $\left| \left \lbrace
        w_i : w_i > j \right\rbrace \right| = k$
  \item we can choose $k$ keys from $\left \lbrace w_i : w_i < j
    \right\rbrace$ in ${{j-1}\choose{k}}$ ways and, for each choice,
    there are $(j-1)!$ ways to sort them;
  \item we can choose $k$ keys from $\left \lbrace w_i : w_i > j
    \right\rbrace$ in ${{n-j}\choose{k}}$ ways and, for each choice,
    there are $(n-j)!$ ways to sort them;
  \item the total number of possible permutation of $n$ keys excluding
    the $pivot$ (which is fixed in the right-most position) is
    $(n-1)!$
  \end{itemize}
  Now we study the mean of the variable $K_j$:
  \begin{displaymath}
    \mathbb{E} \left[ K_j \right] = \sum_{k \geq 0}{k \mathbb{P}\left(
        K_j = k      \right) } 
  \end{displaymath}
  Using ${{n}\choose{m}} =  \frac{n!}{m!(n-m)!} $, we can rewrite:
  \begin{displaymath}
    \mathbb{P}\left(K_j = k \right) =  {{n-j}\choose{k}}
    {{j-1}\choose{k}} \frac{(n-j)! (j-1)! }{(n-1)!} =  {{n-j}\choose{k}}
    {{j-1}\choose{k}} {{n-1}\choose{j-1}}^{-1}
  \end{displaymath}
  We put the previous rewrite of $\mathbb{P}\left(K_j = k \right)$ into
  the definition of $\mathbb{E} \left[ K_j \right]$:
  \begin{displaymath}
    \mathbb{E} \left[ K_j \right] = \sum_{k \geq 0}{k \mathbb{P}\left(
        K_j = k      \right) }  = \sum_{k \geq 0}{k \frac{{{n-j}\choose{k}}
        {{j-1}\choose{k}}}{{{n-1}\choose{j-1}}}}
  \end{displaymath}
  Using the following rewrite for ${{j-1}\choose{k}}$:
  \begin{displaymath}
    {{j-1}\choose{k}} =  \frac{(j-1)!}{k!(j-1-k)!} =
    \frac{(j-1)}{k} \frac{(j-2)!}{(k-1)!(j-1-k)!} =
    \frac{(j-1)}{k}{{j-2}\choose{k-1}} 
  \end{displaymath}
  And ${{n}\choose{m}} = {{n}\choose{n-m}}$ implies ${{j-2}\choose{k-1}}
  = {{j-2}\choose{j-2 -(k-1)}} = {{j-2}\choose{j -k-1}}$, then:
  \begin{displaymath}
    \begin{split}
      \mathbb{E} \left[ K_j \right] &= \sum_{k \geq 0}{k
        \mathbb{P}\left( K_j = k \right) } =
      \frac{1}{{{n-1}\choose{j-1}}} \sum_{k \geq 0}{k
        {{n-j}\choose{k}} {{j-1}\choose{k}}} \\
      &= \frac{1}{{{n-1}\choose{j-1}}} \sum_{k \geq 0}{k
        {{n-j}\choose{k}}  \frac{j-1}{k}{{j-2}\choose{j-k-1}}}
      = \frac{j-1}{{{n-1}\choose{j-1}}} \sum_{k \geq 0}{
        {{n-j}\choose{k}} {{j-2}\choose{j-k-1}}}
    \end{split}
  \end{displaymath}
  Now we recognize the Vandermonde result:
  \begin{displaymath}
    \sum_{k \geq 0}{{{r}\choose{k}} {{s}\choose{n-k}}  } =
    {{r + s}\choose{n}} 
  \end{displaymath}
  which can be proved directly because the ipergeometric distribution
  has exactly the same structure, and being a distribution, it sum up
  to 1. So we use this result applying it to $\sum_{k \geq 0}{
    {{n-j}\choose{k}} {{j-2}\choose{j-k-1}}} = {{n-2}\choose{j-1}} $,
  obtaining:
  \begin{displaymath}
    \begin{split}
      \mathbb{E} \left[ K_j \right] &= \frac{j-1}{{{n-1}\choose{j-1}}}
      {{n-2}\choose{j-1}} =
      \frac{(j-1)(n-2)!(j-1)!(n-j)!}{(n-1)!(j-1)!(n-j-1)!}=\\
      &=\frac{(j-1)(n-2)!(j-1)!(n-j)(n-j-1)!}
      {(n-1)(n-2)!(j-1)!(n-j-1)!}= \frac{(j-1)(n-j)}{n-1} 
    \end{split}
  \end{displaymath}
\end{proof}
Now we proceed with the second proof:
\begin{proof}[Proof of \autoref{eq:quicksort-swaps-two}]
  Let us start with:
  \begin{displaymath}
    \begin{split}
      \frac{1}{n}\sum_{j=1}^{n}{\mathbb{E} \left[K_j \right] } &=
      \frac{1}{n} \sum_{j=1}^{n}{\frac{(j-1)(n-j)}{n-1}} =
      \frac{1}{n(n-1)} \sum_{j=1}^{n}{(j-1)(n-j)}=\\
      &=\frac{1}{n(n-1)} \sum_{j=1}^{n}{(j(1+n)-j^2-n)} \\
      &=\frac{1}{n(n-1)} \left( (n+1)\sum_{j=1}^{n}{j} -
        \sum_{j=1}^{n}{j^2} -n \sum_{j=1}^{n}{1}  \right)\\
      &=\frac{1}{n(n-1)}\left( \frac{n(n+1)^2}{2} -
        \frac{n(n+1)(2n+1)}{6} - n^2 \right) = \ldots =  \frac{n-2}{6} 
    \end{split}
  \end{displaymath} 
\end{proof}
We use \texttt{Maxima} to justify the above calculations:

\noindent
%%%%%%%%%%%%%%%
%%% INPUT:
\begin{minipage}[t]{8ex}{\color{red}\bf
\begin{verbatim}
(%i3) 
\end{verbatim}}
\end{minipage}
\begin{minipage}[t]{\textwidth}{\color{blue}
\begin{verbatim}
expr:(1/(n*(n-1)))* ((n*(n+1)^2)/2 - n*(n+1)*(2*n+1)/6 -n^2);
ratsimp(expr);
\end{verbatim}}
\end{minipage}
%%% OUTPUT:
\definecolor{labelcolor}{RGB}{100,0,0}
\begin{math}\displaystyle
\parbox{8ex}{\color{labelcolor}(\%o3) }
\frac{\frac{n\,{\left( n+1\right) }^{2}}{2}-{n}^{2}-\frac{n\,\left(
      n+1\right) \,\left( 2\,n+1\right) }{6}}{\left( n-1\right) \,n}
\end{math}

\begin{math}\displaystyle
\parbox{8ex}{\color{labelcolor}(\%o4) }
\frac{n-2}{6}
\end{math}
%%%%%%%%%%%%%%%

We are now ready to solve the main recurrence using the same strategy
for the average number of checks, fixing $S(0) = S(1) = S(2) =
0$:
\begin{displaymath}
  \begin{split}
    nS(n) - (n-1)S(n-1) &=  \frac{2n-3}{6} + 2S(n-1)\\
    \frac{S(n)}{n+1} &=  \frac{S(n-1)}{n} +  \frac{2n -3}{6n(n+1)} =
      % \frac{S(2)}{3} +
      \sum_{k=3}^{n}{ \frac{2k-3}{6k(k+1)} }  
  \end{split}
\end{displaymath}
Study the general term of the summation $ \frac{2k-3}{6k(k+1)}$ and
decompose it in partial fractions using Maxima:\\
\noindent
%%%%%%%%%%%%%%%
%%% INPUT:
\begin{minipage}[t]{8ex}{\color{red}\bf
\begin{verbatim}
(%i131) 
\end{verbatim}}
\end{minipage}
\begin{minipage}[t]{\textwidth}{\color{blue}
\begin{verbatim}
generalTerm:(2*k-3)/(6*k*(k-1));
pf:partfrac(generalTerm,k);
integrate(pf , k);
\end{verbatim}}
\end{minipage}
%%% OUTPUT:
\definecolor{labelcolor}{RGB}{100,0,0}
\begin{math}\displaystyle
\parbox{8ex}{\color{labelcolor}(\%o131) }
\frac{2\,k-3}{6\,\left( k-1\right) \,k}
\end{math}\\
\begin{math}\displaystyle
\parbox{8ex}{\color{labelcolor}(\%o132) }
\frac{1}{2\,k}-\frac{1}{6\,\left( k-1\right) }
\end{math}\\
\begin{math}\displaystyle
\parbox{8ex}{\color{labelcolor}(\%o133) }
\frac{\mathrm{log}\left( k\right) }{2}-\frac{\mathrm{log}\left( k-1\right) }{6}
\end{math}
%%%%%%%%%%%%%%%

Hence $S(n) \in O(n logn)$.


\subsection{Final consideration}

We've finished our treatment of sorting algorithms based on checks
between keys. We've seen theoretical results under a probability space
$\Omega$ composed by permutations of $n$ objects. This allow us to
remark that if we setup a simulation where we apply one algorithm seen
above to the entire space $\Omega$ with vectors of a given dimension
$n$, then the average number of checks and swaps must be equal to the
theoretical results obtained in the previous formulas.
