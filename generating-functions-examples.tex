
\section{Examples of generating functions}

In the following sections we study some interesting sequences. Every
section's title shows the sequence under study in that section and its
body contains the building of the relative generating function.

\subsection{$\{c^n\}_{n\in \mathbb{N} }$}
\begin{displaymath}
  \begin{split}
    \mathcal{G} (c^n) = \sum_{n \geq 0}{c^n t^n} &= \sum_{n \geq
      0}{1\cdot (ct)^n} = \sum_{n \geq 0}{a_n \cdot (ct)^n} \quad
    \text{ where } a_n = 1, \forall n \in \mathbb{N} \\
    &= \mathcal{G} (1_n) \circ (ct) = \ \frac{1}{1-ct} \quad \text{ by
    composition of } \mathcal{G} 
  \end{split}
\end{displaymath}
Doing a simple check with Maxima:

\noindent
%%%%%%%%%%%%%%%
%%% INPUT:
\begin{minipage}[t]{8ex}{\color{red}\bf
\begin{verbatim}
(%i226) 
\end{verbatim}}
\end{minipage}
\begin{minipage}[t]{\textwidth}{\color{blue}
\begin{verbatim}
b(t):=1/(1-c*t)$
taylor(b(t),t,0,8);
\end{verbatim}}
\end{minipage}
%%% OUTPUT:
\definecolor{labelcolor}{RGB}{100,0,0}
\begin{math}\displaystyle
\parbox{8ex}{\color{labelcolor}(\%o227) }
1+c\,t+{c}^{2}\,{t}^{2}+{c}^{3}\,{t}^{3}+{c}^{4}\,{t}^{4}+{c}^{5}\,{t}^{5}+{c}^{6}\,{t}^{6}+{c}^{7}\,{t}^{7}+{c}^{8}\,{t}^{8}+...
\end{math}
%%%%%%%%%%%%%%%

\subsection{$\{(-1)^n a_n\}_{n\in \mathbb{N} }$}
\begin{displaymath}
    \mathcal{G} ((-1)^n a_n) = \sum_{n \geq 0}{(-1)^n a_n t^n} = \sum_{n \geq
      0}{a_n (-t)^n} = a(-t)
\end{displaymath}

\subsection{$\{n\}_{n\in \mathbb{N} }$}
\begin{displaymath}
  \mathcal{G} (n) = \mathcal{G} (n\cdot 1_n) = t
  \frac{\partial}{\partial t}\left( \mathcal{G} (1_n) \right) = \frac{t}{{\left( 1-t\right) }^{2}}
  \end{displaymath}
  Doing a little check with Maxima:
  
\noindent
%%%%%%%%%%%%%%%
%%% INPUT:
\begin{minipage}[t]{8ex}{\color{red}\bf
\begin{verbatim}
(%i244) 
\end{verbatim}}
\end{minipage}
\begin{minipage}[t]{\textwidth}{\color{blue}
\begin{verbatim}
b(t):=1/(1-t)$
fg(t):=t*diff(b(t),t);
fg(t);
taylor(fg(t),t,0,8);
\end{verbatim}}
\end{minipage}
%%% OUTPUT:
\definecolor{labelcolor}{RGB}{100,0,0}
\begin{math}\displaystyle
\parbox{8ex}{\color{labelcolor}(\%o245) }
\mathrm{fg}\left( t\right) :=t\,\mathrm{diff}\left( \mathrm{b}\left( t\right) ,t\right) 
\end{math}\\
\begin{math}\displaystyle
\parbox{8ex}{\color{labelcolor}(\%o246) }
\frac{t}{{\left( 1-t\right) }^{2}}
\end{math}\\
\begin{math}\displaystyle
\parbox{8ex}{\color{labelcolor}(\%o247) }
t+2\,{t}^{2}+3\,{t}^{3}+4\,{t}^{4}+5\,{t}^{5}+6\,{t}^{6}+7\,{t}^{7}+8\,{t}^{8}+...
\end{math}
%%%%%%%%%%%%%%%

\subsection{$\{n^2\}_{n\in \mathbb{N} }$}
Using the result of the previous exercise:
\begin{displaymath}
  \mathcal{G} (n^2) = \mathcal{G} (n n) = t
  \frac{\partial}{\partial t}\left( \mathcal{G} (n) \right) = 
  -\frac{{t}^{2}+t}{{t}^{3}-3\,{t}^{2}+3\,t-1}
\end{displaymath}
Just a little check:\\
\noindent
%%%%%%%%%%%%%%%
%%% INPUT:
\begin{minipage}[t]{8ex}{\color{red}\bf
\begin{verbatim}
(%i252) 
\end{verbatim}}
\end{minipage}
\begin{minipage}[t]{\textwidth}{\color{blue}
\begin{verbatim}
snder(t):=t*diff(fg(t),t)$
ratsimp(snder(t));
taylor(snder(t),t,0,8);
\end{verbatim}}
\end{minipage}
%%% OUTPUT:
\definecolor{labelcolor}{RGB}{100,0,0}
\begin{math}\displaystyle
\parbox{8ex}{\color{labelcolor}(\%o253) }
-\frac{{t}^{2}+t}{{t}^{3}-3\,{t}^{2}+3\,t-1}
\end{math}\\
\begin{math}\displaystyle
\parbox{8ex}{\color{labelcolor}(\%o254) }
t+4\,{t}^{2}+9\,{t}^{3}+16\,{t}^{4}+25\,{t}^{5}+36\,{t}^{6}+49\,{t}^{7}+64\,{t}^{8}+...
\end{math}
%%%%%%%%%%%%%%%

\subsection{$\{a_n = 1 \text{ if } n \text{ is even }, 0 \text{
    otherwise}\}_{n\in \mathbb{N} }$}

We compare $\mathcal{G}(a_n) $ with $\mathcal{G} (1_n)$:
\begin{displaymath}
  \begin{split}
    \mathcal{G} (a_n) &= 1 + t^2 + t^4 + t^6 + \ldots \\
    \mathcal{G} (1_n) &= 1 + t^1 + t^2 + t^3 + \ldots \\
  \end{split}
\end{displaymath}
So we can compose $\mathcal{G} (1_n)$ which is known with the function
$t^2$ to obtaing $\mathcal{G} (a_n)$:
\begin{displaymath}
  \mathcal{G} (a_n) =  \frac{1}{1-t} \circ t^2 =  \frac{1}{1-t^2} 
\end{displaymath}
Simple check:

\noindent
%%%%%%%%%%%%%%%
%%% INPUT:
\begin{minipage}[t]{8ex}{\color{red}\bf
\begin{verbatim}
(%i8) 
\end{verbatim}}
\end{minipage}
\begin{minipage}[t]{\textwidth}{\color{blue}
\begin{verbatim}
b(t):=1/(1-t)$
taylor(b(t^2),t,0,10);
\end{verbatim}}
\end{minipage}
%%% OUTPUT:
\definecolor{labelcolor}{RGB}{100,0,0}
\begin{math}\displaystyle
\parbox{8ex}{\color{labelcolor}(\%o9) }
1+{t}^{2}+{t}^{4}+{t}^{6}+{t}^{8}+{t}^{10}+...
\end{math}
%%%%%%%%%%%%%%%

\subsection{$\{a_n = (1,0,0,1,0,0,1,0,0,1,\ldots)\}_{n\in \mathbb{N} }$}

We compare $\mathcal{G}(a_n) $ with $\mathcal{G} (1_n)$ again:
\begin{displaymath}
  \begin{split}
    \mathcal{G} (a_n) &= 1 + t^3 + t^6 + t^9\ldots \\
    \mathcal{G} (1_n) &= 1 + t^1 + t^2 + t^3 + \ldots \\
  \end{split}
\end{displaymath}
So we can compose $\mathcal{G} (1_n)$ which is known with the function
$t^3$ to obtaing $\mathcal{G} (a_n)$:
\begin{displaymath}
  \mathcal{G} (a_n) =  \frac{1}{1-t} \circ t^3 =  \frac{1}{1-t^3} 
\end{displaymath}
Simple check:

\noindent
%%%%%%%%%%%%%%%
%%% INPUT:
\begin{minipage}[t]{8ex}{\color{red}\bf
\begin{verbatim}
(%i10) 
\end{verbatim}}
\end{minipage}
\begin{minipage}[t]{\textwidth}{\color{blue}
\begin{verbatim}
taylor(b(t^3),t,0,10);
\end{verbatim}}
\end{minipage}
%%% OUTPUT:
\definecolor{labelcolor}{RGB}{100,0,0}
\begin{math}\displaystyle
\parbox{8ex}{\color{labelcolor}(\%o10) }
1+{t}^{3}+{t}^{6}+{t}^{9}+...
\end{math}
%%%%%%%%%%%%%%%

\subsection{$\{a_n = (1,0,0,2,0,0,4,0,0,8,\ldots)\}_{n\in \mathbb{N} }$}

We compare $\mathcal{G}(a_n) $ with $\mathcal{G} (2^n)$ again:
\begin{displaymath}
  \begin{split}
    \mathcal{G} (a_n) &= 1 + 2t^3 + 4t^6 + 8t^9\ldots \\
    \mathcal{G} (2^n) &= 1 + 2t^1 + 4t^2 + 8t^3 + \ldots \\
  \end{split}
\end{displaymath}
So we can compose $\mathcal{G} (2^n)$ which is known with the function
$t^3$ to obtaing $\mathcal{G} (a_n)$:
\begin{displaymath}
  \mathcal{G} (a_n) =  \frac{1}{1-2t} \circ t^3 =  \frac{1}{1-2t^3} 
\end{displaymath}
Simple check:

\noindent
%%%%%%%%%%%%%%%
%%% INPUT:
\begin{minipage}[t]{8ex}{\color{red}\bf
\begin{verbatim}
(%i13) 
\end{verbatim}}
\end{minipage}
\begin{minipage}[t]{\textwidth}{\color{blue}
\begin{verbatim}
b(t):=1/(1-2*t)$
taylor(b(t^3),t,0,15);
\end{verbatim}}
\end{minipage}
%%% OUTPUT:
\definecolor{labelcolor}{RGB}{100,0,0}
\begin{math}\displaystyle
\parbox{8ex}{\color{labelcolor}(\%o14) }
1+2\,{t}^{3}+4\,{t}^{6}+8\,{t}^{9}+16\,{t}^{12}+32\,{t}^{15}+...
\end{math}
%%%%%%%%%%%%%%%

\subsection{$\{a_n =  \frac{1}{n} \}_{n\in \mathbb{N} } \wedge a_0 = 0$}

We try to understand the generating function $\mathcal{G}(a_n) $:
\begin{displaymath}
  \begin{split}
    \mathcal{G} ( \frac{1}{n} ) &= t + \frac{1}{2} t^2 +  \frac{1}{3}
    t^3 +  \frac{1}{4} t^4 + \ldots \\
     \frac{a_{n+1}}{a_n} &=  \frac{n}{n+1} \quad \text{ by ratio of
       consecutive terms}\\
      (n+1)a_{n+1}  &= n a_n
  \end{split}
\end{displaymath}
In order to apply the identity principle, it is required to check the
equivalence holds for all $n$. But this isn't the case for $n=0:a_{1}
= 1 \not = 0 = a_0$. So we've to augment the second member with the
term $\delta_{n,k}$ such that $\delta_{n,k} = 1 \text{ if } n=k, 0
\text{ otherwise}$. Augmenting we obtain $(n+1)a_{n+1} = n a_n +
\delta_{n,0}, \forall n \in \mathbb{N} $, in particular for $n=0:a_{1}
= 1 = 1 = a_0 + \delta_{0,0}$. Now it is possible to apply the
identity principle, calling $b_{n+1} = (n+1)a_{n+1}$ (hence $b_n =n
a_n$ and $b_0 = 0$):
\begin{displaymath}
  \begin{split}
    b_{n+1} &= n a_n + \delta_{n,0} \\
    &\downarrow \\
    \mathcal{G} (b_{n+1}) &= \mathcal{G} (n a_n) + \mathcal{G}
    (\delta_{n,0}) \\
    \frac{\mathcal{G} (b_n) - b_0}{t} &= \mathcal{G} (n a_n) +
    \mathcal{G}
    (\delta_{n,0}) \\
    \frac{\mathcal{G} (n a_n)}{t} &= \mathcal{G} (n a_n) + \mathcal{G}
    (\delta_{n,0}) \\
    \frac{t \frac{\partial}{\partial t}\left( \mathcal{G} (a_n)
      \right) }{t} &= \mathcal{G} (n a_n) + \mathcal{G}
    (\delta_{n,0}) \\
    \frac{\partial}{\partial t}\left( \mathcal{G} (a_n) \right) &= t
    \frac{\partial}{\partial t}\left( \mathcal{G} (a_n) \right) + 1 \\
    \frac{\partial}{\partial t}\left( \mathcal{G} (a_n) \right) &=
    \frac{1}{1-t}\\
    \mathcal{G} (a_n) &= \log{ \frac{1}{1-t} }
  \end{split}
\end{displaymath}

Simple check:

\noindent
%%%%%%%%%%%%%%%
%%% INPUT:
\begin{minipage}[t]{8ex}{\color{red}\bf
\begin{verbatim}
(%i15) 
\end{verbatim}}
\end{minipage}
\begin{minipage}[t]{\textwidth}{\color{blue}
\begin{verbatim}
b(t):=log(1/(1-t))$
taylor(b(t),t,0,13);
\end{verbatim}}
\end{minipage}
%%% OUTPUT:
\definecolor{labelcolor}{RGB}{100,0,0}
\begin{math}\displaystyle
\parbox{8ex}{\color{labelcolor}(\%o16) }
t+\frac{{t}^{2}}{2}+\frac{{t}^{3}}{3}+\frac{{t}^{4}}{4}+\frac{{t}^{5}}{5}+\frac{{t}^{6}}{6}+\frac{{t}^{7}}{7}+\frac{{t}^{8}}{8}+\frac{{t}^{9}}{9}+\frac{{t}^{10}}{10}+\frac{{t}^{11}}{11}+\frac{{t}^{12}}{12}+\frac{{t}^{13}}{13}+...
\end{math}
%%%%%%%%%%%%%%%






