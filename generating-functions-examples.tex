
\section{Examples of generating functions}

In the following sections we study some interesting sequences. Every
section's title shows the sequence under study in that section and its
body contains the building of the relative generating function.

\subsection{$\{c^n\}_{n\in \mathbb{N} }$}
\begin{displaymath}
  \begin{split}
    \mathcal{G} (a_n) = \sum_{n \geq 0}{c^n t^n} &= \sum_{n \geq
      0}{1\cdot (ct)^n} = \sum_{n \geq 0}{a_n \cdot (ct)^n} \quad
    \text{ where } a_n = 1, \forall n \in \mathbb{N} \\
    &= \mathcal{G} (1_n) \circ (ct) = \ \frac{1}{1-ct} \quad \text{ by
    composition of } \mathcal{G} 
  \end{split}
\end{displaymath}
Doing a simple check with Maxima:

\noindent
%%%%%%%%%%%%%%%
%%% INPUT:
\begin{minipage}[t]{8ex}{\color{red}\bf
\begin{verbatim}
(%i226) 
\end{verbatim}}
\end{minipage}
\begin{minipage}[t]{\textwidth}{\color{blue}
\begin{verbatim}
b(t):=1/(1-c*t)$
taylor(b(t),t,0,8);
\end{verbatim}}
\end{minipage}
%%% OUTPUT:
\definecolor{labelcolor}{RGB}{100,0,0}
\begin{math}\displaystyle
\parbox{8ex}{\color{labelcolor}(\%o227) }
1+c\,t+{c}^{2}\,{t}^{2}+{c}^{3}\,{t}^{3}+{c}^{4}\,{t}^{4}+{c}^{5}\,{t}^{5}+{c}^{6}\,{t}^{6}+{c}^{7}\,{t}^{7}+{c}^{8}\,{t}^{8}+...
\end{math}
%%%%%%%%%%%%%%%

\subsection{$\{(-1)^n a_n\}_{n\in \mathbb{N} }$}
\begin{displaymath}
    \mathcal{G} (a_n) = \sum_{n \geq 0}{(-1)^n a_n t^n} = \sum_{n \geq
      0}{a_n (-t)^n} = a(-t)
\end{displaymath}

\subsection{$\{n\}_{n\in \mathbb{N} }$}
\begin{displaymath}
  \mathcal{G} (n) = \mathcal{G} (n\cdot 1_n) = t
  \frac{\partial}{\partial t}\left( \mathcal{G} (1_n) \right) = \frac{t}{{\left( 1-t\right) }^{2}}
  \end{displaymath}
  Doing a little check with Maxima:
  
\noindent
%%%%%%%%%%%%%%%
%%% INPUT:
\begin{minipage}[t]{8ex}{\color{red}\bf
\begin{verbatim}
(%i244) 
\end{verbatim}}
\end{minipage}
\begin{minipage}[t]{\textwidth}{\color{blue}
\begin{verbatim}
b(t):=1/(1-t)$
fg(t):=t*diff(b(t),t);
fg(t);
taylor(fg(t),t,0,8);
\end{verbatim}}
\end{minipage}
%%% OUTPUT:
\definecolor{labelcolor}{RGB}{100,0,0}
\begin{math}\displaystyle
\parbox{8ex}{\color{labelcolor}(\%o245) }
\mathrm{fg}\left( t\right) :=t\,\mathrm{diff}\left( \mathrm{b}\left( t\right) ,t\right) 
\end{math}\\
\begin{math}\displaystyle
\parbox{8ex}{\color{labelcolor}(\%o246) }
\frac{t}{{\left( 1-t\right) }^{2}}
\end{math}\\
\begin{math}\displaystyle
\parbox{8ex}{\color{labelcolor}(\%o247) }
t+2\,{t}^{2}+3\,{t}^{3}+4\,{t}^{4}+5\,{t}^{5}+6\,{t}^{6}+7\,{t}^{7}+8\,{t}^{8}+...
\end{math}
%%%%%%%%%%%%%%%

\subsection{$\{n^2\}_{n\in \mathbb{N} }$}
Using the result of the previous exercise:
\begin{displaymath}
  \mathcal{G} (n) = \mathcal{G} (n n) = t
  \frac{\partial}{\partial t}\left( \mathcal{G} (n) \right) = 
  -\frac{{t}^{2}+t}{{t}^{3}-3\,{t}^{2}+3\,t-1}
\end{displaymath}
Just a little check:\\
\noindent
%%%%%%%%%%%%%%%
%%% INPUT:
\begin{minipage}[t]{8ex}{\color{red}\bf
\begin{verbatim}
(%i252) 
\end{verbatim}}
\end{minipage}
\begin{minipage}[t]{\textwidth}{\color{blue}
\begin{verbatim}
snder(t):=t*diff(fg(t),t)$
ratsimp(snder(t));
taylor(snder(t),t,0,8);
\end{verbatim}}
\end{minipage}
%%% OUTPUT:
\definecolor{labelcolor}{RGB}{100,0,0}
\begin{math}\displaystyle
\parbox{8ex}{\color{labelcolor}(\%o253) }
-\frac{{t}^{2}+t}{{t}^{3}-3\,{t}^{2}+3\,t-1}
\end{math}\\
\begin{math}\displaystyle
\parbox{8ex}{\color{labelcolor}(\%o254) }
t+4\,{t}^{2}+9\,{t}^{3}+16\,{t}^{4}+25\,{t}^{5}+36\,{t}^{6}+49\,{t}^{7}+64\,{t}^{8}+...
\end{math}
%%%%%%%%%%%%%%%
