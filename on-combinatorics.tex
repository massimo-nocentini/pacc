
\section{Elements of combinatorics calculus}

\subsection{Permutations $P_n$}

Given a vector $\left ( a_1, \ldots, a_n \right) $ the number of
permutations (or the possible ways to order the elements $a_i$) is
$n!$. A simple proof of that is as follow: for the position 1 are
available $n$ elements, for the second position (after fixing the
first) are available $n-1$ elements arriving to the position $n$ where
only one element is left, hence $n!$.

It is interesting to show how to add a new element $a_{n+1}$ to
existing permutations from the set $\left \lbrace a_1, \ldots,
  a_n\right\rbrace $, this is indeed another proof (by induction) to
the number $n!$ justified above. Having the permutations of length $n$
by hypothesis, we study how is it possible to add $a_{n+1}$ into a
given permutation $w = \left ( a_1, a_2, \ldots, a_n\right) $:
\begin{displaymath}
  \begin{split}
    &\left ( a_{n+1}, a_1, a_2, \ldots, a_n\right)\\
    &\left ( a_1, a_{n+1}, a_2, \ldots, a_n\right)\\
    &\ldots\\
    &\left ( a_1, a_2, \ldots, a_{n+1}, a_n\right)\\
    &\left ( a_1, a_2, \ldots, a_n, a_{n+1}\right)\\    
  \end{split}
\end{displaymath}
Those new vectors are $n+1$ because in the original permutation $w =
\left ( a_1, a_2, \ldots, a_n\right) $ there are $n-1$ ``internal
holes'' between the consecutive elements $a_i$ and $a_{i+1}$, plus the
left-most and right-most ``external holes'' $(\Box, a_1,...)$ and
$(..., a_n, \Box)$. So far we've considered only one permutation $w$
of $n$ elements, repeating the same reasoning for the others $w_i$
(which are $n!-1$), it must exists $n+1$ new vectors for each of
them. So the number of permutation of $n+1$ elements is $(n+1)n! =
(n+1)!$ (what we would have expected).

\subsection{Dispositions $D_{n,k}$}

A special case of permutations are dispositions, that is the number of
different ways in which $k$ elements can be choosen from a set of
length $n$, with $n \geq k$. We have $n$ elements available for the
first choice, $n-1$ for the second choice arriving to $n-k+1$ elements
for the $k$-th choice, hence $D_{n,k} = n(n-1)\cdots(n-k+1)$.  We can
write an example to show how to build dispositions. Let $\Omega =
\left \lbrace a,b,c, d\right\rbrace $ and we want to build $D_{4,2}$,
that is the set of ordered pairs, paying attention that $ \forall i:
(i,i)\not \in D_{n,2}$. The following pairs are the candidates (the
pairs of the form $(i,i)$ aren't listed because after we choose $i$
for the first position it isn't possible to choose it again for the
second):
\begin{displaymath}
  \begin{split}
    \left ( a, b \right)\quad
    \left ( a, c \right)\quad
    \left ( a, d \right)\\
    \left ( b, a \right)\quad
    \left ( b, c \right)\quad
    \left ( b, d \right)\\
    \left ( c, a \right)\quad
    \left ( c, b \right)\quad
    \left ( c, d \right)\\
    \left ( d, a \right)\quad
    \left ( d, b \right)\quad
    \left ( d, c \right)\\
  \end{split}
\end{displaymath}
We've build the previous matrix because we take care about
the order in which the pairs are built. In the next section we'll see
that there exists another set that doesn't take care of the order.

\subsection{Combinations $C_{n,k}$}

A special case of dispositions are combinations. With combinations,
notation $C_{n,k}$, we mean a set of sets $\left \lbrace \left \lbrace
    a_{i_1}, \ldots, a_{i_k} \right\rbrace \right\rbrace$ without
regard to the order in which $a_{i_1}, \ldots, a_{i_k}$ appear. We can
see combinations like dispositions modulo the symmetric relation:
$(j,i) \in C_{n,2} \rightarrow (i, j) \not \in C_{n,2}$ and $(i,i)\not
\in C_{n,2}, \forall i$). The following submatrix is to combinations
as the previous one is to dispositions:
\begin{displaymath}
  \begin{split}
    \left ( a, b \right)\quad
    \left ( a, c \right)\quad
    \left ( a, d \right)\\
    \quad
    \left ( b, c \right)\quad
    \left ( b, d \right)\\
    \quad
    \quad
    \left ( c, d \right)\\
    \quad
    \quad
    \\
  \end{split}
\end{displaymath}
The combinations matrix is obtained by the disposition matrix, broking
the symmetry, only half of them belongs to the dispositions
$C_{4,2}$. Hence $C_{n,k} = {{n}\choose{k}} = \frac{D_{n,k}}{k!} =
\frac{n!}{k!(n-k)!)} $.
