\section{Generating functions basics}

In this section we approach the basics for the generating functions
topic.

\subsection{Approaching}

The main goal of this section is to manipulate a sequence $\{a_n\}_{n\in
  \mathbb{N} }$ with a function $a(t)$ such that:
\begin{displaymath}
  a(t) = \sum_{n \geq 0}{a_n t^n} 
\end{displaymath}
Using $a(t)$ allow us to let $t$ be unbounded and we wont assign any
value to it (it is just a ``placeholder'').

We introduce an operator $\mathcal{G}$ which consume a sequence
$\{a_n\}_{n\in \mathbb{N} }$ and produce a function $a(t)$ as defined
above. In the following examples we experiment a little with Maxima,
which implement $\mathcal{G}$ with the function 
\texttt{\color{blue} ggf(s)}, for some sequence $s$.\\

Let $\{a_n = n\}$:\\
\noindent
%%%%%%%%%%%%%%%
%%% INPUT:
\begin{minipage}[t]{8ex}{\color{red}\bf
\begin{verbatim}
(%i109) 
\end{verbatim}}
\end{minipage}
\begin{minipage}[t]{\textwidth}{\color{blue}
\begin{verbatim}
sequence:makelist(n,n,0, 6);
taylor(ggf(sequence), x, 0, 8);
\end{verbatim}}
\end{minipage}
%%% OUTPUT:
\definecolor{labelcolor}{RGB}{100,0,0}
\begin{math}\displaystyle
\parbox{8ex}{\color{labelcolor}(\%o109) }
[0,1,2,3,4,5,6]
\end{math}\\
\begin{math}\displaystyle
\parbox{8ex}{\color{labelcolor}(\%o110) }
x+2\,{x}^{2}+3\,{x}^{3}+4\,{x}^{4}+5\,{x}^{5}+6\,{x}^{6}+...
\end{math}\\
%%%%%%%%%%%%%%%
So $\mathcal{G}(a_n) = a(x)=
x+2\,{x}^{2}+3\,{x}^{3}+4\,{x}^{4}+5\,{x}^{5}+6\,{x}^{6}+...$\\

Let $\{a_n = 1 \text{ if } n \text{ is even }, 0 \text{ otherwise}\}$:\\
\noindent
%%%%%%%%%%%%%%%
%%% INPUT:
\begin{minipage}[t]{8ex}{\color{red}\bf
\begin{verbatim}
(%i105) 
\end{verbatim}}
\end{minipage}
\begin{minipage}[t]{\textwidth}{\color{blue}
\begin{verbatim}
onlyEvenTerms(n):=if is(mod (n,2)=0) then 1 else 0$
sequence:makelist(onlyEvenTerms(n),n,0, 6);
ggfOnlyEvenTerms:ggf(sequence);
taylor(ggfOnlyEvenTerms, x, 0, 12);
\end{verbatim}}
\end{minipage}
%%% OUTPUT:
\definecolor{labelcolor}{RGB}{100,0,0}
\begin{math}\displaystyle
\parbox{8ex}{\color{labelcolor}(\%o106) }
[1,0,1,0,1,0,1]
\end{math}\\
\begin{math}\displaystyle
\parbox{8ex}{\color{labelcolor}(\%o107) }
-\frac{1}{{x}^{2}-1}
\end{math}\\
\begin{math}\displaystyle
\parbox{8ex}{\color{labelcolor}(\%o108) }
1+{x}^{2}+{x}^{4}+{x}^{6}+{x}^{8}+{x}^{10}+{x}^{12}+...
\end{math}\\
%%%%%%%%%%%%%%%
So $\mathcal{G}(a_n) =
a(x)=1+{x}^{2}+{x}^{4}+{x}^{6}+{x}^{8}+{x}^{10}+{x}^{12}+...$.\\

Let $\{a_n = 1 \text{ if } n \leq 2,  0 \text{ otherwise}\}$:\\
\noindent
%%%%%%%%%%%%%%%
%%% INPUT:
\begin{minipage}[t]{8ex}{\color{red}\bf
\begin{verbatim}
(%i97) 
\end{verbatim}}
\end{minipage}
\begin{minipage}[t]{\textwidth}{\color{blue}
\begin{verbatim}
onlyFirstThreeTerms(n):=if is(n <= 2) then 1 else 0$
sequence:makelist(onlyFirstThreeTerms(n),n,0, 6);
ggfOnlyFirstThreeTerms:ggf(sequence);
taylor(ggfOnlyFirstThreeTerms, x, 0, 100);
\end{verbatim}}
\end{minipage}
%%% OUTPUT:
\definecolor{labelcolor}{RGB}{100,0,0}
\begin{math}\displaystyle
\parbox{8ex}{\color{labelcolor}(\%o98) }
[1,1,1,0,0,0,0]
\end{math}\\
\begin{math}\displaystyle
\parbox{8ex}{\color{labelcolor}(\%o99) }
{x}^{2}+x+1
\end{math}\\
\begin{math}\displaystyle
\parbox{8ex}{\color{labelcolor}(\%o100) }
1+x+{x}^{2}+...
\end{math}\\
%%%%%%%%%%%%%%%
So $\mathcal{G}(a_n) =a(x)=1+{x}+{x}^{2}$.\\

\subsection{Linearity of $\mathcal{G} $}
Let $\{a_n\}_{n\in\mathbb{N} }$ and $\{b_n\}_{n\in\mathbb{N} } $ be
two sequences, $a(t)$ and $b(t)$ their formal power series
respectively. Lets sum both of them:
\begin{displaymath}
  \begin{split}
    \mathcal{G} (a_n)+ \mathcal{G} (b_n) &= a_0 + a_1 t
    + a_2 t^2 + \ldots + a_n t^n + \ldots \\
    &+ b_0 + b_1 t + b_2 t^2 + \ldots + b_n t^n + \ldots =\\
    &= (a_0 + b_0) + (a_1+b_1) t + (a_2 + b_2) t^2 + \ldots +
    (a_n+b_n) t^n + \ldots =\\
    &= \mathcal{G}(a_n + b_n) 
  \end{split}
\end{displaymath}
It is possible to generalize, obtaining:
\begin{displaymath}
  \begin{split}
    \alpha\mathcal{G}(a_n) &= \mathcal{G}(\alpha a_n)\\
    \alpha\mathcal{G}(a_n) +  \beta\mathcal{G}(b_n) &=
    \mathcal{G}(\alpha a_n + \beta b_n) 
  \end{split}
\end{displaymath}
So $\mathcal{G}$ is a \emph{linear operator}.

\subsection{Convolution of $\mathcal{G} $}

Let $\{a_n\}_{n\in\mathbb{N} }$ and $\{b_n\}_{n\in\mathbb{N} } $ be
two sequences, $a(t)$ and $b(t)$ their formal power series
respectively. We study the product $a(t)b(t)$ with Maxima:\\
\noindent
%%%%%%%%%%%%%%%
%%% INPUT:
\begin{minipage}[t]{8ex}{\color{red}\bf
\begin{verbatim}
(%i190) 
\end{verbatim}}
\end{minipage}
\begin{minipage}[t]{\textwidth}{\color{blue}
\begin{verbatim}
makeSequence(t,j):=sum (j[i]*t^i, i, 0, inf);
as:taylor(makeSequence(t,a), t, 0, 3);
bs:taylor(makeSequence(t,b), t, 0, 3);
as*bs;
sum ((sum(a[k]*b[n-k],k,0,n))*t^n, n, 0, inf);
taylor(sum((sum(a[k]*b[n-k],k,0,n))*t^n,n,0,inf), t, 0, 3);
\end{verbatim}}
\end{minipage}
%%% OUTPUT:
\definecolor{labelcolor}{RGB}{100,0,0}
\begin{math}\displaystyle
\parbox{8ex}{\color{labelcolor}(\%o190) }
\mathrm{makeSequence}\left( t,j\right) :=\sum_{i=0}^{\infty }{j}_{i}\,{t}^{i}
\end{math}\\
\begin{math}\displaystyle
\parbox{8ex}{\color{labelcolor}(\%o191) }
{a}_{0}+{a}_{1}\,t+{a}_{2}\,{t}^{2}+{a}_{3}\,{t}^{3}+...
\end{math}\\
\begin{math}\displaystyle
\parbox{8ex}{\color{labelcolor}(\%o192) }
{b}_{0}+{b}_{1}\,t+{b}_{2}\,{t}^{2}+{b}_{3}\,{t}^{3}+...
\end{math}\\
\begin{math}\displaystyle
  \parbox{8ex}{\color{labelcolor}(\%o193) } {a}_{0}\,{b}_{0}+\left(
    {a}_{1}\,{b}_{0}+{a}_{0}\,{b}_{1}\right) \,t+\left(
    {a}_{2}\,{b}_{0}+{a}_{1}\,{b}_{1}+{a}_{0}\,{b}_{2}\right)
  \,{t}^{2}+\\
  \left(
    {a}_{3}\,{b}_{0}+{a}_{2}\,{b}_{1}+{a}_{1}\,{b}_{2}+{a}_{0}\,{b}_{3}\right)
  \,{t}^{3}+...
\end{math}\\
\begin{math}\displaystyle
\parbox{8ex}{\color{labelcolor}(\%o194) }
\sum_{n=0}^{\infty }\left( \sum_{k=0}^{n}{a}_{k}\,{b}_{n-k}\right) \,{t}^{n}
\end{math}\\
\begin{math}\displaystyle
\parbox{8ex}{\color{labelcolor}(\%o195) }
\left( \sum_{k=0}^{0}{b}_{-k}\,{a}_{k}\right) +\left(
  \sum_{k=0}^{1}{b}_{1-k}\,{a}_{k}\right) \,t+\left(
  \sum_{k=0}^{2}{b}_{2-k}\,{a}_{k}\right) \,{t}^{2}+\\
\left( \sum_{k=0}^{3}{b}_{3-k}\,{a}_{k}\right) \,{t}^{3}+...
\end{math}\\
%%%%%%%%%%%%%%%
So $\mathcal{G}(a_n)\mathcal{G}(b_n) =\mathcal{G}\left(
  \sum_{k=0}^{n}{a}_{k}\,{b}_{n-k}\right)$.






