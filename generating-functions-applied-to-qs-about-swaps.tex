\section{Generating function of quicksort's average swaps }

In this section we go back again to the recurrence about the average
number of swaps for the basic implementation of quicksort algorithm
and we'll try to find the generating function for the sequence $S_n$
of average swaps.

We start again from the known recurrence:
\begin{displaymath}
  \begin{split}
    S_n &= \frac{n-2}{6} + \frac{2}{n}\sum_{j=0}^{n-1}{S_j}\\
    6nS_n &= n(n-2) + 12\sum_{j=0}^{n-1}{S_j}\\
  \end{split}
\end{displaymath}
with $S_0 = S_1 = 0$. We've to remember to check if the recurrence
holds for all $n$, the previous doesn't for $n=1$ because $0 \not=
-1$. So we fix augmenting the second member with the sequence
$\delta_{n,1}$ which returns $1$ if and only if $n=1$, now the
recurrence holds for all $n \in \mathbb{N} $. By the \emph{Identity
  principle} we get:
\begin{displaymath}
  \begin{split}
    6 \mathcal{G} (nS_n) &= \mathcal{G} (n^2) -2 \mathcal{G} (n) + 12
    \sum_{j=0}^{n-1}{S_j} + \mathcal{G} (\delta_{n,1}) \\
    6t S^{\prime} (t) &= \frac{t + t^2}{(1-t)^3} - \frac{2t}{(1-t)^2}
    + 12 t \mathcal{G} \left(\sum_{j=0}^{n}{S_j}\right) + t\\
    6t S^{\prime} (t) &= \frac{t + t^2 -2t(1-t) + t(1-t)^3}{(1-t)^3} +
    \frac{12 t}{1-t} S(t)\\
    6 S^{\prime} (t) &= \frac{1 + t -2(1-t) + (1-t)^3}{(1-t)^3} +
    \frac{12}{1-t} S(t)\\
    S^{\prime} (t) &= \frac{t^2(3-t)}{6(1-t)^3} +
    \frac{2}{1-t} S(t)\\
  \end{split}
\end{displaymath}
with initial condition $S(0) = 0$. Now we study the homogeneous
associate equation $\mu(t)$ (excluding for a moment the term $
\frac{t^2(3-t)}{6(1-t)^3}$):
\begin{eqnarray}
  \mu^{\prime} (t) &=&  \frac{2}{1-t} \mu (t)\\
  \frac{\mu^{\prime}(t)}{\mu (t)} &=& \frac{2}{1-t} \label{eq:assoc-swaps-qs}\\
  \log\mu (t) &=& -2\log (1-t) = \log \frac{1}{(1-t)^2} \quad \text{
    integrating both members} \\
  \label{eq:final-assoc-swaps-qs}
  \mu (t) &=& \frac{1}{(1-t)^2}\quad \text{
    exping both members}
\end{eqnarray}
Now we study the derivative which complies the $\mu$ function,
solution of the unknown function $S(t)$ we hope to find:
\begin{displaymath}
  \begin{split}
    \frac{\partial}{\partial t}\left( \frac{S(t)}{\mu (t)} \right) &=
    \frac{S^{\prime} (t)\mu (t) - S(t)\mu^{\prime} (t)}{\mu (t)^2} \\
    &= \mu (t)\frac{S^{\prime} (t) - S(t) \frac{\mu^{\prime} (t)}{\mu
        (t)} }{\mu (t)^2} \\
    &= \frac{1}{\mu (t)} (S^{\prime} (t) - \frac{2}{1-t}S(t) ) \quad
    \text{by \autoref{eq:assoc-swaps-qs}}\\
    &= \frac{1}{\mu (t)}\frac{t^2(3-t)}{6(1-t)^3} \\
    &= \frac{t^2(3-t)}{6(1-t)}
  \end{split}
\end{displaymath}
We do simple manipulation in order to ease the integration:
\begin{displaymath}
  \begin{split}
    \frac{\partial}{\partial t}\left( \frac{S(t)}{\mu (t)} \right) &=
    \frac{t^2(3-t)}{6(1-t)} = \frac{t^2(1-t + 2)}{6(1-t)} =
    \frac{1}{6}\left(t^2 + \frac{2t^2}{1-t} \right) =
    \frac{1}{6}\left(t^2 + \frac{2t^2-2+2}{1-t} \right) =\\
    &= \frac{1}{6}\left(t^2 + \frac{2(t-1)(t+1)}{1-t} +
      \frac{2}{1-t}\right) = \frac{1}{6}\left(t^2 -
      \frac{2(1-t)(t+1)}{1-t} + \frac{2}{1-t}\right) =\\
    &=
    \frac{1}{6}\left(t^2 - 2(t+1) + \frac{2}{1-t}\right)
  \end{split}
\end{displaymath}
Now we integrate both members:
\begin{displaymath}
  \begin{split}
    \frac{S(t)}{\mu (t)} &= \frac{1}{6}\left( \frac{t^3}{3} - t^2 -2t
      -2\log{(1-t)} + k\right)\\
    S(t) &= \frac{1-t}{t^2(3-t)}\left( \frac{t^3}{3} -
      t^2 -2t
      -2\log{(1-t)} + k\right)\\
  \end{split}
\end{displaymath}
After simplification and $k=0$ by the initial conditions $S(0)=S(1)=0$
we get:
\begin{equation}
  S(t) =  \frac{1}{3 (1-t)^2} \log{  \frac{1}{1-t} } +  \frac{t^3
    -3t^2 -6t}{18 (1-t)^2} 
\end{equation}

Simple check with Maxima:

\noindent
%%%%%%%%%%%%%%%
%%% INPUT:
\begin{minipage}[t]{8ex}{\color{red}\bf
\begin{verbatim}
(%i3) 
\end{verbatim}}
\end{minipage}
\begin{minipage}[t]{\textwidth}{\color{blue}
\begin{verbatim}
s(t):=(1/(3*(1-t)^2))*log(1/(1-t)) + (t^3-3*t^2-6*t)/(18*(1-t)^2)$
taylor(s(t),t,0,8);
\end{verbatim}}
\end{minipage}
%%% OUTPUT:
\definecolor{labelcolor}{RGB}{100,0,0}
\begin{math}\displaystyle
\parbox{8ex}{\color{labelcolor}(\%o4) }
\frac{{t}^{3}}{6}+\frac{5\,{t}^{4}}{12}+\frac{11\,{t}^{5}}{15}+\frac{199\,{t}^{6}}{180}+\frac{961\,{t}^{7}}{630}+\frac{1669\,{t}^{8}}{840}+...
\end{math}
%%%%%%%%%%%%%%%









