
\section{Generating function quicksort's average checks }
In this section we go back to the recurrence about the average number
of checks for the basic implementation of quicksort algorithm and
we'll try to find the generating function for the sequence $C_n$ of
average checks.

We start again from the known recurrence:
\begin{displaymath}
  \begin{split}
    C_n &= n+1 +  \frac{1}{n}\sum_{k=1}^{n}{C_k + C_{n-k}}  \\
    C_n &= n+1 +  \frac{2}{n}\sum_{k=0}^{n-1}{C_k}  \\
    nC_n &= n(n+1) + 2\sum_{k=0}^{n-1}{C_k}  \\
    \mathcal{G} (nC_n) &= \mathcal{G} (n^2+n) +
    2\mathcal{G} \left(\sum_{k=0}^{n-1}{C_k} \right) \\
    \mathcal{G} (nC_n) &= \mathcal{G} (n^2) +\mathcal{G} (n) +
    2\mathcal{G} \left(\sum_{k=0}^{n-1}{C_k} \right) \\
  \end{split}
\end{displaymath}
Now we remeber the following equalities:
\begin{displaymath}
  \begin{split}
    \mathcal{G} (n) &=  \frac{t}{(1-t)^2} \\
    \mathcal{G} (n^2) &=
    t\frac{\partial}{\partial t}(\mathcal{G} (n)) =
    \frac{t(1+t)}{(1-t)^3} \\
  \end{split}
\end{displaymath}
And it is possible to see $\sum_{k=0}^{n-1}{C_k} = a_{n-1}$ as a
sequence, so $\mathcal{G} (a_{n-1}) = t \mathcal{G} (a_n)$;
substituting those rewritings and using the convolution property we
get:
\begin{displaymath}
  \begin{split}
    t \frac{\partial}{\partial t}\left( \mathcal{G} (C_n) \right) &=
    \frac{t}{(1-t)^2} + \frac{t(1+t)}{(1-t)^3} + 2t\mathcal{G}
    \left(\sum_{k=0}^{n}{C_k} \right)\\
    t \frac{\partial}{\partial t}\left( \mathcal{G} (C_n) \right) &=
    \frac{t -t^2+ t+t^2}{(1-t)^3} + \frac{2t}{1-t} \mathcal{G}
    \left(C_n \right) \quad \text{by } \mathcal{G}
    \left(\sum_{k=0}^{n}{1_k C_k} \right)=\mathcal{G} (1_n)\mathcal{G} (C_n)\\
    \frac{\partial}{\partial t}\left( \mathcal{G} (C_n) \right) &=
    \frac{2}{(1-t)^3} + \frac{2}{1-t} \mathcal{G}
    \left(C_n \right)\\
  \end{split}
\end{displaymath}
We can write the same expression by $\mathcal{G} (C_n) = C(t)$ as a
differential equation:
\begin{equation}
  \label{eq:main-assoc-checks-qs}
  C^{\prime}(t) = \frac{2}{(1-t)^3} + \frac{2}{1-t} C(t)
\end{equation}
with initial condition $C(0) = 0$. Now we study the homogeneous
associate equation $\mu(t)$ (excluding for a moment the term $
\frac{2}{(1-t)^3}$):
\begin{eqnarray}
    \mu^{\prime} (t) &=&  \frac{2}{1-t} \mu (t)\\
    \frac{\mu^{\prime}(t)}{\mu (t)} &=& \frac{2}{1-t} \label{eq:assoc-checks-qs}\\
    \log\mu (t) &=& -2\log (1-t) = \log \frac{1}{(1-t)^2} \quad \text{
      integrating both members} \\
    \label{eq:final-assoc-checks-qs}
    \mu (t) &=& \frac{1}{(1-t)^2}\quad \text{
      exping both members}
\end{eqnarray}
Now we study the derivative which complies the $\mu$ function,
solution of the unknown function $C(t)$ we hope to find:
\begin{displaymath}
  \begin{split}
    \frac{\partial}{\partial t}\left( \frac{C(t)}{\mu (t)} \right) &=
    \frac{C^{\prime} (t)\mu (t) - C(t)\mu^{\prime} (t)}{\mu (t)^2} \\
    &= \mu (t)\frac{C^{\prime} (t) - C(t) \frac{\mu^{\prime} (t)}{\mu
        (t)} }{\mu (t)^2} \\
    &= \frac{1}{\mu (t)} (C^{\prime} (t) - \frac{2}{1-t}C(t) ) \quad
    \text{by \autoref{eq:assoc-checks-qs}}\\
    &= \frac{1}{\mu (t)}\frac{2}{(1-t)^3}\quad \text{by
      \autoref{eq:main-assoc-checks-qs}} \\
    &= \frac{2}{1-t}
  \end{split}
\end{displaymath}
Now we integrate both members:
\begin{displaymath}
  \begin{split}
    \frac{C(t)}{\mu (t)} &= -2\log (1-t) +k \\
    C(t) &= \frac{2\log{ \frac{1}{1-t} } + k}{(1-t)^2}\quad \text{by
      \autoref{eq:final-assoc-checks-qs}}
  \end{split}
\end{displaymath}
Since $C(0)=0$ it must have $k=0$, hence:
\begin{equation}
  C(t) = \frac{2}{(1-t)^2}\log{ \frac{1}{1-t} }
\end{equation}

Simple check with Maxima:

\noindent
%%%%%%%%%%%%%%%
%%% INPUT:
\begin{minipage}[t]{8ex}{\color{red}\bf
\begin{verbatim}
(%i5) 
\end{verbatim}}
\end{minipage}
\begin{minipage}[t]{\textwidth}{\color{blue}
\begin{verbatim}
c(t):=(2/(1-t)^2)*log(1/(1-t))$
taylor(c(t),t,0,8);
\end{verbatim}}
\end{minipage}
%%% OUTPUT:
\definecolor{labelcolor}{RGB}{100,0,0}
\begin{math}\displaystyle
\parbox{8ex}{\color{labelcolor}(\%o6)}
2\,t+5\,{t}^{2}+\frac{26\,{t}^{3}}{3}+\frac{77\,{t}^{4}}{6}+\frac{87\,{t}^{5}}{5}+\frac{223\,{t}^{6}}{10}+\frac{962\,{t}^{7}}{35}+\frac{4609\,{t}^{8}}{140}+...
\end{math}
%%%%%%%%%%%%%%%







 




