
\section{Inverse of generating function operator}

\subsection{Some properties}

\subsection{Netwon's rule}

\subsection{Inverting Fibonacci generating function}
In this section we'll study the extract coefficient operator $[t^{n}]$
applied to the generating function of Fibonacci's sequence.

In \autoref{eq:fibonacci-generating-function} we found the generating
function for Fibonacci's sequence, in order to apply the Newton's
rule, we have to reach something like $(1 + \alpha t)^r$, for some
$\alpha, r \in \mathbb{R} $. So the idea is to write the term
$\frac{t}{1-t-t^2}$ as partial fractions and try to apply $[t^{n}]$.

First we solve $1-t-t^2 = 0$ respect $t$, having two solutions
$\phi=\frac{\sqrt{5}-1}{2}$ and $\hat{\phi}=-\frac{\sqrt{5}+1}{2}$
(Maxima results \parbox{8ex}{\color{labelcolor}(\%o759) }
and \parbox{8ex}{\color{labelcolor}(\%o760) }).

Those solutions are very interesting and enjoy the following
properties (Maxima results \parbox{8ex}{\color{labelcolor}(\%o761) }
and \parbox{8ex}{\color{labelcolor}(\%o762) }):
\begin{displaymath}
  \begin{split}
    \phi + \hat{\phi} &= \frac{\sqrt{5}}{2} - \frac{1}{2} -
    \frac{\sqrt{5}}{2} - \frac{1}{2} = -1\\
    \phi \hat{\phi} &= \frac{(\sqrt{5} -1)(-\sqrt{5} -1)}{4} =
    \frac{-5+1}{4} = -1\\
    &\downarrow\\
    \phi + \hat{\phi} &= \phi \hat{\phi} = -1
  \end{split}
\end{displaymath}
Now we use the previous solutions to factor $1-t-t^2$:
\begin{displaymath}
  \begin{split}
    1-t-t^2 &= -\left( t-\phi \right)\left( t-\hat{\phi} \right)\\
    &= -(-\phi)(-\hat{\phi})\left(1 - \frac{t}{\phi} \right)
    \left(1 - \frac{t}{\hat{\phi}} \right)\\
    &= -\phi\hat{\phi}\left(1 - \frac{t}{\phi} \right)
    \left(1 - \frac{t}{\hat{\phi}} \right)\\
    &= \left(1 - \frac{t}{\phi} \right) \left(1 - \frac{t}{\hat{\phi}}
    \right) \quad \text{ by above properties}
  \end{split}
\end{displaymath}
Now we'll find two constants $A, B \in \mathbb{R} $ to build the
partial fractions:
\begin{displaymath}
  \begin{split}
    \frac{t}{1-t-t^2} &= \frac{A}{\left(1 - \frac{t}{\phi} \right)} +
    \frac{B}{ \left(1 - \frac{t}{\hat{\phi}} \right)}= \frac{A \left(1
        - \frac{t}{\hat{\phi}} \right) + B\left(1 - \frac{t}{\phi}
      \right)}{\left(1 - \frac{t}{\phi} \right) \left(1 -
        \frac{t}{\hat{\phi}} \right)} = \frac{(A+B) -t
      \left(\frac{A}{\hat{\phi}} +\frac{B}{\phi}
      \right)}{\left(1 - \frac{t}{\phi} \right) \left(1 -
        \frac{t}{\hat{\phi}} \right)}
  \end{split}
\end{displaymath}
So we've to solve the system with two equations in two unknowns:
$A+B=0$ and $\left(\frac{A}{\hat{\phi}} +\frac{B}{\phi} \right) = -1$,
which yields $A= \frac{1}{\sqrt{5}} $ and $B= -\frac{1}{\sqrt{5}}$ (as
confirmed in \parbox{8ex}{\color{labelcolor}(\%o765) }). So we've
found the partial fractions decomposition (as confirmed
in \parbox{8ex}{\color{labelcolor}(\%o767) }):
\begin{equation}
  \frac{t}{1-t-t^2} =  \frac{1}{\sqrt{5}}\frac{1}{1-\frac{t}{\phi}}-
  \frac{1}{\sqrt{5}}\frac{1}{1-\frac{t}{\hat{\phi}}}
\end{equation}
We apply the $[t^{n}] $ to both members:
\begin{displaymath}
  \begin{split}
    [t^{n}]\frac{t}{1-t-t^2} &=
    \frac{1}{\sqrt{5}}[t^{n}]\frac{1}{1-\frac{t}{\phi}}-
    \frac{1}{\sqrt{5}}[t^{n}]\frac{1}{1-\frac{t}{\hat{\phi}}}\\
    &= \frac{1}{\sqrt{5}}[t^{n}]\left(1-\frac{t}{\phi}\right)^{-1}-
    \frac{1}{\sqrt{5}}[t^{n}]\left(1-\frac{t}{\hat{\phi}}\right)^{-1}
    \\
    &=
    \frac{1}{\sqrt{5}}{{-1}\choose{n}}\left(-\frac{1}{\phi}\right)^{n}-
    \frac{1}{\sqrt{5}}{{-1}\choose{n}}\left(-\frac{1}{\hat{\phi}}\right)^{n}
    \quad \text{by Newton's rule}\\
    &= \frac{1}{\sqrt{5}}(-1)^n(-1)^n\left(\frac{1}{\phi}\right)^{n}-
    \frac{1}{\sqrt{5}}(-1)^n(-1)^n\left(\frac{1}{\hat{\phi}}\right)^{n}\\
    & \quad \text{by } {{-1}\choose{n}} = (-1)^n{{1+n-1}\choose{n}} =
    (-1)^n\\
    &= \frac{1}{\sqrt{5}}\left(\frac{1}{\phi}\right)^{n}-
    \frac{1}{\sqrt{5}}\left(\frac{1}{\hat{\phi}}\right)^{n}\\
  \end{split}
\end{displaymath}
Checking with Maxima, in \parbox{8ex}{\color{labelcolor}(\%o770) } we
obtain the right Fibonacci numbers.




\noindent
%%%%%%%%%%%%%%%
%%% INPUT:
\begin{minipage}[t]{8ex}{\color{red}\bf
\begin{verbatim}
(%i757) 
\end{verbatim}}
\end{minipage}
\begin{minipage}[t]{\textwidth}{\color{blue}
\begin{verbatim}
eqFib:1-k-k^2=0;
solutions:solve(eqFib,k);
phi:ev(k,solutions[2]);
phiHat:ev(k,solutions[1]);
ratsimp(phi*phiHat);
ratsimp(phi+phiHat);
ratsimp(-phi*phiHat*(1-t/phi)*(1-t/phiHat));
ratsimp((1+phi*t)*(1+phiHat*t));
ABvalues:solve([A+B=0, (A/phiHat)+(B/phi)=-1],[A,B]);
parFracAB(A,B):=(A/(1-t/phi))+(B/(1-t/phiHat));
ratsimp(ev(parFracAB(A,B),ABvalues));
coef(n,A,B):=A*(1/phi)^n + B*(1/phiHat)^n;
coefCurry(n):=ratsimp(ev(coef(n,A,B),ABvalues))$
map(coefCurry, makelist(n,n,0,10));
\end{verbatim}}
\end{minipage}
%%% OUTPUT:
\definecolor{labelcolor}{RGB}{100,0,0}
\begin{math}\displaystyle
\parbox{8ex}{\color{labelcolor}(\%o757) }
-{k}^{2}-k+1=0
\end{math}

\begin{math}\displaystyle
\parbox{8ex}{\color{labelcolor}(\%o758) }
[k=-\frac{\sqrt{5}+1}{2},k=\frac{\sqrt{5}-1}{2}]
\end{math}

\begin{math}\displaystyle
\parbox{8ex}{\color{labelcolor}(\%o759) }
\frac{\sqrt{5}-1}{2}
\end{math}

\begin{math}\displaystyle
\parbox{8ex}{\color{labelcolor}(\%o760) }
-\frac{\sqrt{5}+1}{2}
\end{math}

\begin{math}\displaystyle
\parbox{8ex}{\color{labelcolor}(\%o761) }
-1
\end{math}

\begin{math}\displaystyle
\parbox{8ex}{\color{labelcolor}(\%o762) }
-1
\end{math}

\begin{math}\displaystyle
\parbox{8ex}{\color{labelcolor}(\%o763) }
-{t}^{2}-t+1
\end{math}

\begin{math}\displaystyle
\parbox{8ex}{\color{labelcolor}(\%o764) }
-{t}^{2}-t+1
\end{math}

\begin{math}\displaystyle
\parbox{8ex}{\color{labelcolor}(\%o765) }
[[A=\frac{1}{\sqrt{5}},B=-\frac{1}{\sqrt{5}}]]
\end{math}

\begin{math}\displaystyle
\parbox{8ex}{\color{labelcolor}(\%o766) }
\mathrm{parFracAB}\left( A,B\right)
:=\frac{A}{1-\frac{t}{\phi}}+\frac{B}{1-\frac{t}{phiHat}}
\end{math}

\begin{math}\displaystyle
\parbox{8ex}{\color{labelcolor}(\%o767) }
-\frac{t}{{t}^{2}+t-1}
\end{math}

\begin{math}\displaystyle
\parbox{8ex}{\color{labelcolor}(\%o768) }
\mathrm{coef}\left( n,A,B\right) :=A\,{\left( \frac{1}{\phi}\right)
}^{n}+B\,{\left( \frac{1}{phiHat}\right) }^{n}
\end{math}

\begin{math}\displaystyle
\parbox{8ex}{\color{labelcolor}(\%o770) }
[0,1,1,2,3,5,8,13,21,34,55]
\end{math}
%%%%%%%%%%%%%%%


