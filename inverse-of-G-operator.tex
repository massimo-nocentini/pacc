
\section{ $[t^{n}] $ inverse operator of generating function operator}

In this section we'll study the inverse operator $[t^{n}] $ of
$\mathcal{G} $, which consume a generating function $g(t)$, where
$g(t) = \mathcal{G} (\{a_n\}_{n\in\mathbb{N} } )$, and returns the
$a_n$ coefficient of the sequence $\{a_n\}_{n\in\mathbb{N} }$ (in
other words, $[t^n]a(t) = f(n)$, so $[t^n]$ returns a function of
$n$), formally $[t^{n}] g(t) = a_n$. It is interesting to note that
the function $g(t)$ to which $[t^n]$ is applied to depends only on $t$
(in the last lectures we will see a special case of application
$[t^n]g(t,n)$).

\subsection{Linearity of $[t^{n}] $}
\begin{displaymath}
  \begin{split}
    [t^{n}] (\alpha a(t) + \beta b(t)) = [t^{n}]
    \left(\sum_{n=0}^{\infty}{(\alpha a_n + \beta b_n )t^n}\right) =
    \alpha a_n + \beta b_n
  \end{split}
\end{displaymath}

\subsection{Translation of $[t^{n}] $}
Let $s\in \mathbb{Z}$:
\begin{displaymath}
  \begin{split}
    [t^{n}] t^s a(t) &= [t^{n}] t^s (a_0 + a_1 t + a_2 t^2 + \ldots +
    a_n t^n + \ldots )\\
    &= [t^{n}] (a_0 t^s + a_1 t^{s+1} + a_2 t^{s+2} + \ldots + a_n
    t^{n+s} + \ldots )\\
    &= [t^{n-s}] a(t)
  \end{split}
\end{displaymath}
A simple observation:
\begin{displaymath}
  [t^{n}]  \frac{  a(t)}{t^s} = [t^{n+s}] a(t)
\end{displaymath}

\subsection{Derivation of $[t^{n}] $}
\begin{displaymath}
  \begin{split}
    a(t) &= a_0 + a_1 t + a_2 t^2 + \ldots + a_n t^n + \ldots \\
    \frac{\partial}{\partial t}\left(a(t) \right) &= a_1 + 2a_2 t +
    3a_3 t^2+\ldots + (n+1)a_{n+1} t^n + \ldots \\
    [t^{n}] \frac{\partial}{\partial t}\left(a(t) \right) &=
    (n+1)a_{n+1}
  \end{split}
\end{displaymath}
It is interesting to note the following fact:
\begin{displaymath}
  \begin{split}
    [t^{n}] \frac{\partial}{\partial t}\left(a(t) \right) &=
    (n+1)a_{n+1}\\
    &\downarrow \\
    [t^{n-1}] \frac{\partial}{\partial t}\left(a(t) \right) &= n a_{n} \\
    n[t^{n}]a(t) &= n a_{n}\quad \text{by def of } [t^{n}]
  \end{split}
\end{displaymath}
So the following holds:
\begin{equation}
  \label{eq:implication-der-property-of-inverse-of-G}
  [t^{n}]a(t) =  \frac{1}{n} [t^{n-1}] a^{\prime}(t)
\end{equation}


\subsection{Convolution of $[t^{n}] $}
\begin{displaymath}
  \begin{split}
    [t^{n}] a(t)b(t) &= [t^{n}] \mathcal{G} (a_n) \mathcal{G} (b_n) =
    [t^{n}]\mathcal{G} \left(\sum_{k=0}^{n}{a_k b_{n-k}} \right) =
    \sum_{k=0}^{n}{a_k b_{n-k}}
  \end{split}
\end{displaymath}

\subsection{Composition of $[t^{n}] $}
\begin{displaymath}
  \begin{split}
    [t^{n}] a(t)\circ b(t) &=[t^{n}] a(b(t))
    =[t^{n}]\sum_{k=0}^{\infty}{a_k (\mathcal{G} (b_n))^k}\\
    &=\sum_{k=0}^{\infty}{a_k [t^{n}] b(t)^k} \quad \text{by linearity
      of } [t^n]
  \end{split}
\end{displaymath}

\subsection{Netwon's rule}
Let $\alpha, r \in \mathbb{R} $:
\begin{equation}
  \label{eq:newton-rule}
  [t^{n}] (1 + \alpha t)^r = {{r}\choose{n}} \alpha^n
\end{equation}
\begin{proof}[Proof]
  We repeatedly apply
  \autoref{eq:implication-der-property-of-inverse-of-G} to $[t^{n}] (1
  + \alpha t)^r$ getting the following:
  \begin{displaymath}
    \begin{split}
      [t^{n}] (1 + \alpha t)^r &= \frac{1}{n} [t^{n-1}] r\alpha(1 +
      \alpha t)^{r-1} \\
      &= \frac{r\alpha}{n} \frac{1}{n-1}[t^{n-2}] (r-1)\alpha(1 +
      \alpha t)^{r-2}\\
      &= \frac{r\alpha}{n}
      \frac{(r-1)\alpha}{n-1}\frac{1}{n-2}[t^{n-3}] (r-2)\alpha(1 +
      \alpha t)^{r-3}\\
      &\vdots\\
      &= \frac{r\alpha}{n} \frac{(r-1)\alpha}{n-1}\cdots
      \frac{(r-n+2))\alpha}{n-n+2} \frac{1}{n-n+1}[t^{n-n}]
      (r-n+1)\alpha(1 +
      \alpha t)^{r-n}\\
      &= \frac{r\alpha}{n} \frac{(r-1)\alpha}{n-1}\cdots
      \frac{(r-n+2))\alpha}{n-n+2} \frac{(r-n+1)\alpha}{n-n+1}[t^{0}]
      (1 +      \alpha t)^{r-n}\\
      &= \alpha^n \frac{r(r-1)\cdots(r-n+1)}{n!}[t^{0}]
      (1 +      \alpha t)^{r-n}\\
      &= \alpha^n {{r}\choose{n}} [t^{0}]
      (1 +      \alpha t)^{r-n}\\
      &= \alpha^n {{r}\choose{n}} 
    \end{split}
  \end{displaymath}
  where $[t^{0}] (1 + \alpha t)^{r-n} = 1$ because, let $f(t)= (1 +
  \mu t)^{\nu}$ in:
  \begin{displaymath}
    [t^0]f(t) = [t^0] \left(\sum_{k=0}^{\infty}{
        \frac{f^k(0)}{k!}t^k } \right) = f^0(0) = f(0) = 1 \quad
    \forall \mu, \nu \in \mathbb{R} 
\end{displaymath}

\end{proof}

\subsection{Inverting Fibonacci generating function}
In this section we'll study the extract coefficient operator $[t^{n}]$
applied to the generating function of Fibonacci's sequence.

In \autoref{eq:fibonacci-generating-function} we found the generating
function for Fibonacci's sequence, in order to apply the Newton's
rule, we have to reach something like $(1 + \alpha t)^r$, for some
$\alpha, r \in \mathbb{R} $. So the idea is to write the term
$\frac{t}{1-t-t^2}$ as partial fractions and try to apply $[t^{n}]$.

First we solve $1-t-t^2 = 0$ respect $t$, having two solutions
$\phi=\frac{\sqrt{5}-1}{2}$ and $\hat{\phi}=-\frac{\sqrt{5}+1}{2}$
(Maxima results \parbox{8ex}{\color{labelcolor}(\%o759) }
and \parbox{8ex}{\color{labelcolor}(\%o760) }).

Those solutions are very interesting and enjoy the following
properties (Maxima results \parbox{8ex}{\color{labelcolor}(\%o761) }
and \parbox{8ex}{\color{labelcolor}(\%o762) }):
\begin{displaymath}
  \begin{split}
    \phi + \hat{\phi} &= \frac{\sqrt{5}}{2} - \frac{1}{2} -
    \frac{\sqrt{5}}{2} - \frac{1}{2} = -1\\
    \phi \hat{\phi} &= \frac{(\sqrt{5} -1)(-\sqrt{5} -1)}{4} =
    \frac{-5+1}{4} = -1\\
    &\downarrow\\
    \phi + \hat{\phi} &= \phi \hat{\phi} = -1
  \end{split}
\end{displaymath}
Now we use the previous solutions to factor $1-t-t^2$:
\begin{displaymath}
  \begin{split}
    1-t-t^2 &= -\left( t-\phi \right)\left( t-\hat{\phi} \right)\\
    &= -(-\phi)(-\hat{\phi})\left(1 - \frac{t}{\phi} \right)
    \left(1 - \frac{t}{\hat{\phi}} \right)\\
    &= -\phi\hat{\phi}\left(1 - \frac{t}{\phi} \right)
    \left(1 - \frac{t}{\hat{\phi}} \right)\\
    &= \left(1 - \frac{t}{\phi} \right) \left(1 - \frac{t}{\hat{\phi}}
    \right) \quad \text{ by above properties}
  \end{split}
\end{displaymath}
Now we'll find two constants $A, B \in \mathbb{R} $ to build the
partial fractions:
\begin{displaymath}
  \begin{split}
    \frac{t}{1-t-t^2} &= \frac{A}{\left(1 - \frac{t}{\phi} \right)} +
    \frac{B}{ \left(1 - \frac{t}{\hat{\phi}} \right)}= \frac{A \left(1
        - \frac{t}{\hat{\phi}} \right) + B\left(1 - \frac{t}{\phi}
      \right)}{\left(1 - \frac{t}{\phi} \right) \left(1 -
        \frac{t}{\hat{\phi}} \right)} = \frac{(A+B) -t
      \left(\frac{A}{\hat{\phi}} +\frac{B}{\phi}
      \right)}{\left(1 - \frac{t}{\phi} \right) \left(1 -
        \frac{t}{\hat{\phi}} \right)}
  \end{split}
\end{displaymath}
So we've to solve the system with two equations in two unknowns:
$A+B=0$ and $\left(\frac{A}{\hat{\phi}} +\frac{B}{\phi} \right) = -1$,
which yields $A= \frac{1}{\sqrt{5}} $ and $B= -\frac{1}{\sqrt{5}}$ (as
confirmed in \parbox{8ex}{\color{labelcolor}(\%o765) }). So we've
found the partial fractions decomposition (as confirmed
in \parbox{8ex}{\color{labelcolor}(\%o767) }):
\begin{equation}
  \frac{t}{1-t-t^2} =  \frac{1}{\sqrt{5}}\frac{1}{1-\frac{t}{\phi}}-
  \frac{1}{\sqrt{5}}\frac{1}{1-\frac{t}{\hat{\phi}}}
\end{equation}
We apply the $[t^{n}] $ to both members:
\begin{displaymath}
  \begin{split}
    [t^{n}]\frac{t}{1-t-t^2} &=
    \frac{1}{\sqrt{5}}[t^{n}]\frac{1}{1-\frac{t}{\phi}}-
    \frac{1}{\sqrt{5}}[t^{n}]\frac{1}{1-\frac{t}{\hat{\phi}}}\\
    &= \frac{1}{\sqrt{5}}[t^{n}]\left(1-\frac{t}{\phi}\right)^{-1}-
    \frac{1}{\sqrt{5}}[t^{n}]\left(1-\frac{t}{\hat{\phi}}\right)^{-1}
    \\
    &=
    \frac{1}{\sqrt{5}}{{-1}\choose{n}}\left(-\frac{1}{\phi}\right)^{n}-
    \frac{1}{\sqrt{5}}{{-1}\choose{n}}\left(-\frac{1}{\hat{\phi}}\right)^{n}
    \quad \text{by Newton's rule}\\
    &= \frac{1}{\sqrt{5}}(-1)^n(-1)^n\left(\frac{1}{\phi}\right)^{n}-
    \frac{1}{\sqrt{5}}(-1)^n(-1)^n\left(\frac{1}{\hat{\phi}}\right)^{n}\\
    & \quad \text{by } {{-1}\choose{n}} = (-1)^n{{1+n-1}\choose{n}} =
    (-1)^n\\
    &= \frac{1}{\sqrt{5}}\left(\frac{1}{\phi}\right)^{n}-
    \frac{1}{\sqrt{5}}\left(\frac{1}{\hat{\phi}}\right)^{n}\\
  \end{split}
\end{displaymath}
Checking with Maxima, in \parbox{8ex}{\color{labelcolor}(\%o770) } we
obtain the right Fibonacci numbers.




\noindent
%%%%%%%%%%%%%%%
%%% INPUT:
\begin{minipage}[t]{8ex}{\color{red}\bf
\begin{verbatim}
(%i757) 
\end{verbatim}}
\end{minipage}
\begin{minipage}[t]{\textwidth}{\color{blue}
\begin{verbatim}
eqFib:1-k-k^2=0;
solutions:solve(eqFib,k);
phi:ev(k,solutions[2]);
phiHat:ev(k,solutions[1]);
ratsimp(phi*phiHat);
ratsimp(phi+phiHat);
ratsimp(-phi*phiHat*(1-t/phi)*(1-t/phiHat));
ratsimp((1+phi*t)*(1+phiHat*t));
ABvalues:solve([A+B=0, (A/phiHat)+(B/phi)=-1],[A,B]);
parFracAB(A,B):=(A/(1-t/phi))+(B/(1-t/phiHat));
ratsimp(ev(parFracAB(A,B),ABvalues));
coef(n,A,B):=A*(1/phi)^n + B*(1/phiHat)^n;
coefCurry(n):=ratsimp(ev(coef(n,A,B),ABvalues))$
map(coefCurry, makelist(n,n,0,10));
\end{verbatim}}
\end{minipage}
%%% OUTPUT:
\definecolor{labelcolor}{RGB}{100,0,0}
\begin{math}\displaystyle
\parbox{8ex}{\color{labelcolor}(\%o757) }
-{k}^{2}-k+1=0
\end{math}

\begin{math}\displaystyle
\parbox{8ex}{\color{labelcolor}(\%o758) }
[k=-\frac{\sqrt{5}+1}{2},k=\frac{\sqrt{5}-1}{2}]
\end{math}

\begin{math}\displaystyle
\parbox{8ex}{\color{labelcolor}(\%o759) }
\frac{\sqrt{5}-1}{2}
\end{math}

\begin{math}\displaystyle
\parbox{8ex}{\color{labelcolor}(\%o760) }
-\frac{\sqrt{5}+1}{2}
\end{math}

\begin{math}\displaystyle
\parbox{8ex}{\color{labelcolor}(\%o761) }
-1
\end{math}

\begin{math}\displaystyle
\parbox{8ex}{\color{labelcolor}(\%o762) }
-1
\end{math}

\begin{math}\displaystyle
\parbox{8ex}{\color{labelcolor}(\%o763) }
-{t}^{2}-t+1
\end{math}

\begin{math}\displaystyle
\parbox{8ex}{\color{labelcolor}(\%o764) }
-{t}^{2}-t+1
\end{math}

\begin{math}\displaystyle
\parbox{8ex}{\color{labelcolor}(\%o765) }
[[A=\frac{1}{\sqrt{5}},B=-\frac{1}{\sqrt{5}}]]
\end{math}

\begin{math}\displaystyle
\parbox{8ex}{\color{labelcolor}(\%o766) }
\mathrm{parFracAB}\left( A,B\right)
:=\frac{A}{1-\frac{t}{\phi}}+\frac{B}{1-\frac{t}{phiHat}}
\end{math}

\begin{math}\displaystyle
\parbox{8ex}{\color{labelcolor}(\%o767) }
-\frac{t}{{t}^{2}+t-1}
\end{math}

\begin{math}\displaystyle
\parbox{8ex}{\color{labelcolor}(\%o768) }
\mathrm{coef}\left( n,A,B\right) :=A\,{\left( \frac{1}{\phi}\right)
}^{n}+B\,{\left( \frac{1}{phiHat}\right) }^{n}
\end{math}

\begin{math}\displaystyle
\parbox{8ex}{\color{labelcolor}(\%o770) }
[0,1,1,2,3,5,8,13,21,34,55]
\end{math}
%%%%%%%%%%%%%%%


